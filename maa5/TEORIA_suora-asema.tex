\laatikko{
KIRJOITA TÄHÄN LUKUUN

\luettelo{
§ sama kulmakerroin --> yhdensuuntaiset tai sama suora
§ eri kulmakerroin --> yksi leikkauspiste, kytkentä yhtälöpareihin
§ suoran ja normaalin kulmakertoimet, $k_1k_2=-1$.
}

KIITOS!}

Piirretään koordinaatistoon kaksi eri suoraa ja tutkitaan niiden leikkauspisteitä.
On kaksi eri vaihtoehtoa:
\numerointi{
 § Suorat eivät leikkaa.
 § Suorat leikkaavat täsmälleen yhdessä pisteessä.
}

Ensimmäisessä tapauksessa suorilla on sama kulmakerroin. Sanotaan, että ne ovat yhdensuuntaiset.

\begin{kuva}
    kuvaaja.pohja(-3, 3, -3, 3, nimiX = "$x$", nimiY = "$y$")
    kuvaaja.piirra("2*x+1", nimi = "$y=2x+1$")
    kuvaaja.piirra("2*x-3", nimi = "$y=2x-3$")
\end{kuva}

Jos suorat ovat pystysuoria, ei niiden kulmakerrointa ole määritelty. Myös tällöin suorat ovat yhdensuuntaisia.

\begin{kuva}
    kuvaaja.pohja(-3, 3, -2, 2, nimiX = "$x$", nimiY = "$y$")
    kuvaaja.piirraParametri("1", "t", -2, 2, nimi = "$x=-1$")
    kuvaaja.piirraParametri("-2", "t", -2, 2, nimi = "$x=2$")
\end{kuva}

\laatikko{
Jos kahdella suoralla on sama kulmakerroin tai molemmat suorat ovat pystysuoria, ovat suorat yhdensuuntaiset.
}

Jos suorilla on eri kulmakerroin, on niillä täsmälleen yksi leikkauspiste. 

\begin{kuva}
    kuvaaja.pohja(-3, 3, -3, 3, nimiX = "$x$", nimiY = "$y$")
    kuvaaja.piirra("2*x+1", nimi = "$y=2x+1$")
    kuvaaja.piirra("-3*x+4", nimi = "$y=-3x+4$")
\end{kuva}

Kulmakertoimien perusteella voidaan myös päätellä, ovatko suorat toisiaan vastaan kohtisuorassa.

\begin{kuva}
    kuvaaja.pohja(-2, 4, -2, 4, nimiX = "$x$", nimiY = "$y$")
    kuvaaja.piirra("2*x+1", nimi = "$y=2x+1$")
    kuvaaja.piirra("-0.5*x+2", nimi = "$y=-(1/2)x+2$")
\end{kuva}

(Tähän kuvaan voisi hahmotella kolmiot, joiden avulla suorien kulmakertoimet nähdään kuvasta.)

\laatikko{
Kaksi suoraa ovat toisiaan vastaan kohtisuorassa täsmälleen silloin, jos niiden kulmakertoimien tulo on $-1$
}
