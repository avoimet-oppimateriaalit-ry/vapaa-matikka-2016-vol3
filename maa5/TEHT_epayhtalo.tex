\begin{tehtavasivu}

\subsubsection*{Opi perusteet}

\begin{tehtava}
Esitä ilman itseisarvomerkkejä.
	\alakohdat{
		§ $|x|<6$
		§ $|x|>10$
		§ $|x|\leq 1,6$
		§ $|x|\neq 64$
	}
	\begin{vastaus}
		\alakohdat{
			§ $-6<x<6$
			§ $x<-10$ tai $x>10$
			§ $-1,6\leq x\leq 1,6$
			§ $x\neq -64$ tai $x\leq 64$
		}
	\end{vastaus}
\end{tehtava}

\begin{tehtava}
Ratkaise seuraavat epäyhtälöt.
	\alakohdat{
		§ $|x+6|>3$
		§ $|x-5|<2$
		§ $|x-7|\leq 1$
		§ $-|x+2|> -2$
	}
	\begin{vastaus}
		\alakohdat{
			§ $x<-9$ tai $x>-3$
			§ $3<x<7$
			§ $6\leq x\leq 8$
			§ $-4 < x < 0$
		}
	\end{vastaus}
\end{tehtava}

\begin{tehtava}
Ratkaise seuraavat epäyhtälöt.
	\alakohdat{
		§ $|2x-6|>x$
		§ $|x+2|<-x+1$
		§ $|3-x|\geq -2x+4$
	}
	\begin{vastaus}
		\alakohdat{
			§ $x<2$ tai $x>4$
			§ $x>\frac{1}{2}$
			§ $x\geq 1$
		}
	\end{vastaus}
\end{tehtava}

\begin{tehtava}
	Ratkaise epäyhtälö $|x^2+1| \ge 3$.
	\begin{vastaus}
		$x<-\sqrt{2}$ tai $x>\sqrt{2}$
	\end{vastaus}
\end{tehtava}

\subsubsection*{Hallitse kokonaisuus}

\begin{tehtava}
	Ratkaise epäyhtälö $|x+a| \leq |x|+|a|$ vapaan parametrin $a$ funktiona.
	\begin{vastaus}
		$x \in \rr$ (tulos tunnetaan kolmioepäyhtälönä)
	\end{vastaus}
\end{tehtava}

\begin{tehtava}
Ratkaise seuraavat epäyhtälöt.
	\alakohdat{
		§ $|x+1|<|3x-3|$
		§ $|22+x| \geq |5x+2|$
		§ $|12x-7| \leq -3x+8$
	}
	\begin{vastaus}
		\alakohdat{
			§ $x< \frac{1}{2}$ tai $x>2$
			§ $-4 \leq x \leq 5$
			§ $-\frac{1}{9} \leq x \leq 1$
		}
	\end{vastaus}
\end{tehtava}

\begin{tehtava}
Millä vakion $a$ arvoilla epäyhtälö on tosi, kun $x=3$?
	\alakohdat{
		§ $|2x+a|>|7|$
		§ $|12-x| > |ax+5a|$
	}
	\begin{vastaus}
		\alakohdat{
			§ $a< -13$ tai $a>1$
			§ $-\frac{9}{8} < a < \frac{9}{8}$
		}
	\end{vastaus}
\end{tehtava}

%Ankkatehtävä ihan hauska tyyliltään mutta ainakin itselleni hieman vaikeaselkoinen. Voisikohan tätä selkiyttää toisin sanamuodoin?
\begin{tehtava}
	Ankka haluaa asettua lukusuoralle pisteeseen $-x$ ja piilottaa rahansa toiseen pisteeseen. Pisteessä $3$ asuu karhu, joka ryöstää ankan rahat, elleivät rahat ole lähempänä ankkaa kuin karhua. Toisinaan ankan pitää käydä ostoksilla pisteessä $0$, ja se haluaa ottaa rahat mukaansa kuljettuaan 2/3 matkasta. Mitä luvun $x$ on oltava, jotta tämä on mahdollista?
	\begin{vastaus}
		$-3<x<9$
	\end{vastaus}
\end{tehtava}

\begin{tehtava}
	Todista, että
	\begin{align*}
		|a| \geq |b| \Leftrightarrow a^2 \geq b^2 \Leftrightarrow (a-b)(a+b) \geq 0
	\end{align*}
	\begin{vastaus}
		Vinkki: Huomaa, että $|a|^2 = a^2$. Mitä voidaan sanoa lausekkeesta $|a|^2-|b|^2$ muistikaavan avulla?
	\end{vastaus}
\end{tehtava}

\subsubsection*{Sekalaisia tehtäviä}

TÄHÄN TEHTÄVIÄ SIJOITTAMISTA ODOTTAMAAN

\end{tehtavasivu}
