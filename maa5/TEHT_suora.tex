\begin{tehtavasivu}

\subsubsection*{Opi perusteet}

\begin{tehtava}
	\alakohdat{
		§ Määritä suoran $y=-2x+5$ ja $y$-akselin leikkauspiste.
		§ Määritä suoran $y=3x+1$ ja $x$-akselin leikkauspiste.
	}
\begin{vastaus}
	\alakohdat{
		§ $(0, 5)$
		§ $(-\frac{1}{3}, 0)$
	}
\end{vastaus}
\end{tehtava}

\begin{tehtava}
Ratkaise suorien $y=10x$ ja $y=x-9$ leikkauspiste.
\begin{vastaus}
% http://www.wolframalpha.com/input/?i=y%3D-10x+y%3Dx-9
$(-1, -10)$
\end{vastaus}
\end{tehtava}

\begin{tehtava}
Ratkaise suorien $y=-5x+3$ ja $y=2x-17$ leikkauspiste.
\begin{vastaus}
% http://www.wolframalpha.com/input/?i=y%3D-5x%2B3%2C+y%3D2x-17
$(\frac{20}{7}, -\frac{79}{70})$
\end{vastaus}
\end{tehtava}

\begin{tehtava}
Mikä on $x$-akselin suuntaisen suoran, joka kulkee pisteen $(1, 3)$ kautta, yhtälö?
\begin{vastaus}
$y=3$
\end{vastaus}
\end{tehtava}

\begin{tehtava}
Mikä on suoran $y=3,14x-10$ yhtälön
\alakohdat{
§ vakiotermi,
§ kulmakerroin?
}
\begin{vastaus}
a)$-10$ b) $3,14$
\end{vastaus}
\end{tehtava}

\begin{tehtava}
Suora kulkee pisteiden $(2, 1)$ ja $(5, 9)$ kautta. Määritä suoran kulmakerroin.
\begin{vastaus}
Kulmakerroin on $\frac{8}{3}$
\end{vastaus}
\end{tehtava}

\begin{tehtava}
Piirrä suora $y=9x-1$.
\begin{vastaus}
puuttuu
\end{vastaus}
\end{tehtava}

\subsubsection*{Hallitse kokonaisuus}

\begin{tehtava}
Ratkaise suorien $y=-x+2$ ja $y=2x-4$ leikkauspiste.
\begin{vastaus}
$(2, 0)$
\end{vastaus}
\end{tehtava}

\begin{tehtava}
Määritä
\alakohdat{
§ $x$-akselin suuntaisen suoran,
§ $y$-akselin suuntaisen suoran kulmakerroin?
}
\begin{vastaus}
a) $0$ b) ei määritelty %(ääretön)
\end{vastaus}
\end{tehtava}

\begin{tehtava}
Piirrä suora $y=-2x+3$.
\begin{vastaus}
puuttuu
\end{vastaus}
\end{tehtava}

\begin{tehtava}
Määritä suoran $\frac{y}{2}=\frac{x}{2}+2$ ja $x$-akselin leikkauspiste.
\begin{vastaus}
$(-4, 0)$
\end{vastaus}
\end{tehtava}

\begin{tehtava}
Ratkaise suoran $6=-60x+600y$ ja $x$-akselin leikkauspiste.
\begin{vastaus}
$x=-\frac{1}{10}$
\end{vastaus}
\end{tehtava}

\begin{tehtava}
Missä pistessä suora $y=\frac{16x}{25}+\frac{36}{49}$
\alakohdat{
§ leikkaa $x$-akselin,
§ leikkaa $y$-akselin?
}
\begin{vastaus}
a)$(-\frac{225}{196}, 0)$ b) $(0, \frac{36}{49})$
\end{vastaus}
\end{tehtava}

\begin{tehtava}
Ratkaise suorien $y=-\frac{2}{5}$ ja $3y=18x+20$ leikkauspiste.
\begin{vastaus}
$(-\frac{53}{45}, -\frac{2}{5})$
\end{vastaus}
\end{tehtava}

\begin{tehtava}
Ratkaise suoran $16y-9x=-5y-11x+27$ ja $x$-akselin leikkauspiste.
\begin{vastaus}
$(11, 0)$
\end{vastaus}
\end{tehtava}

\subsubsection*{Sekalaisia tehtäviä}

LAITA TEHTÄVÄT TÄHÄN, JOS ET OLE VARMA VAIKEUSASTEESTA TAI TEHTÄVÄ
EI TÄLLÄ HETKELLÄ SOVI MUKAAN

\end{tehtavasivu}
