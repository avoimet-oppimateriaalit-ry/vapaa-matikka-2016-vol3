\begin{tehtavasivu}

\sarjaA % Opi perusteet

\begin{tehtava}
Mikä on suoran yhtälö ratkaistussa muodossa?
\alakohdat{
§ $3x + y -5 = 0$
§ $-15x + 5y + 20 =0$
§ $2x - 3y + 4 = 0$
}
\begin{vastaus}
\alakohdat{
§ $y=-3x+5$
§ $y=3x-4$
§ $y=\frac{2x}{3} + \frac{4}{3}$
}
\end{vastaus}
\end{tehtava}

\begin{tehtava}
Mikä on suoran yhtälö normaalimuodossa?
\alakohdat{
§ $y=-15x+2$
§ $2y=11x+7$
§ $2y+5x-8=13y-6x-8$
}
\begin{vastaus}
\alakohdat{
§ $15x+y-2=0$
§ $11x-2y+7=0$
§ $-11x+11y=0$
}
\end{vastaus}
\end{tehtava}

\begin{tehtava}
Suoran kulmakerroin on $\frac{1}{2}$ ja suora kulkee pisteen
\alakohdat{
§ $(-12, 4)$
§ $(3, 9)$
}
kautta. Mikä on suoran yhtälö?
\begin{vastaus}
\alakohdat{
§ $y=\frac{1}{2}x+10$
§ $y=\frac{1}{2}x+\frac{15}{2}$
}
\end{vastaus}
\end{tehtava}

\begin{tehtava}
Suora kohtaa $x$-akselin, kun $x=15$, ja suora kulkee pisteen
\alakohdat{
§ $(-13, 1)$
§ $(7, 8)$
}
kautta. Mikä on suoran yhtälö?
\begin{vastaus}
\alakohdat{
§ $y=\frac{-1}{28}x+\frac{15}{28}$
§ $y=-x+15$
}
\end{vastaus}
\end{tehtava}

\begin{tehtava}
Suora kohtaa $y$-akselin, kun $y=10$, ja suora kulkee pisteen
\alakohdat{
§ $(-12, 2)$
§ $(4, 14)$
}
kautta. Mikä on suoran yhtälö?
\begin{vastaus}
\alakohdat{
§ $y=\frac{2}{3}x+10$
§ $y=x+10$
}
\end{vastaus}
\end{tehtava}

\begin{tehtava}
Mikä on pisteiden
\alakohdat{
§ $(1, -2)$ ja $(3, 1)$
§ $(0, 0)$ ja $(-4, 4)$ kautta kulkevan suoran yhtälö?
}
\begin{vastaus}
\alakohdat{
§ $y=\frac{3}{2}x-\frac{7}{2}$
§ $y=-x$
}
\end{vastaus}
\end{tehtava}

\begin {tehtava}
Suora kulkee pisteiden $(3, 4)$ ja $(\sqrt{3}, 1)$ kautta. Määritä suoran kulmakerroin.
\begin {vastaus}
$\frac{\sqrt{3}-1}{\sqrt{3}}$
\end {vastaus}
\end {tehtava}

\sarjaB % Hallitse kokonaisuus

\begin{tehtava}
Tutki ovatko pisteet  
\alakohdat{
§ $(1, -5)$, $(4, -23)$ja $(4, -239)$
§ $(7, 3)$, $(-2, 10)$ ja $(-3, 90)$ samalla suoralla?
}
\begin{vastaus}
\alakohdat{
§ kyllä
§ ei
}
\end{vastaus}
\end{tehtava}

\begin{tehtava}
Määritä luku $t$ niin, että pisteet $(-t+3, -4)$, $(6, t-5)$ ja $(5, -4)$ ovat samalla suoralla.
\begin{vastaus}
$t=-2$ tai $t=1$
\end{vastaus}
\end{tehtava}

\begin{tehtava}
Voidaan osoittaa, että tason pisteet $A$, $B$ ja $C$ ovat samalla suoralla jos ja vain jos jokin pisteiden välisistä etäisyyksistä, $AB$, $BC$ tai $CA$, on kahden muun summa. (Jos suoralta kiinnitetään ensin kaksi pistettä, tämä voidaan ottaa myös suoran määritelmäksi.) Ovatko pisteet samalla suoralla
	\alakohdat{
		§ $(13,5)$, $(20,0)$ ja $(7,9)$
		§ $(-10,-8)$, $(15,22)$ ja $(5,10)$?
	}
	\begin{vastaus}
		\alakohdat{
			§ Eivät
			§ Ovat
		}
	\end{vastaus}
\end{tehtava}

\sarjaC % Syvennä osaamistasi

\begin{tehtava}
Suora kulkee pisteiden $(\frac{1}{r+4}, r^2)$ ja $(\frac{1}{r+4}+1, 16)$ kautta, $r \neq -4$. Millä $y$:n arvoilla piste $(1,y)$ on suoralla?
	\begin{vastaus}
	$y = r+12$
	\end{vastaus}
\end{tehtava}

\sarjaD % Sekalaisia tehtäviä

TÄHÄN TEHTÄVIÄ SIJOITTAMISTA ODOTTAMAAN

\end{tehtavasivu}
