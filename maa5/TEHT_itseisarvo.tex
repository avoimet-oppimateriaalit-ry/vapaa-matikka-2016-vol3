\begin{tehtavasivu}

TÄHÄN TEHTÄVIÄ SIJOITTAMISTA ODOTTAMAAN

\begin{tehtava}
	Esitä lauseke ilman itseisarvomerkkejä.
	\alakohdat{
		§ $|\pi-2^2|$
		§ $|2x-6|$
		§ $x+|6-3x|$
	}
	\begin{vastaus}
		\alakohdat{
			§ $-(\pi-4)=-\pi+4=4-\pi$
			§ $\begin{cases}
					-2x+6, & \jos x<3 \\
					2x-6, & \jos x \geq 3
				\end{cases}$
			§ $\begin{cases}
					x+(6-3x), & \jos 6-3x \geq0 \\
					x-(6-3x), & \jos 6-3x <0 
				\end{cases}
				=\begin{cases}
					x+6-3x, & \jos -3x \geq-6 \\
					x-6+3x, & \jos -3x <-6 
				\end{cases}
				=\begin{cases}
					-2x+6, & \jos x \leq2 \\
					4x-6, & \jos x >2 
				\end{cases}$
		}
	\end{vastaus}
\end{tehtava}

\begin{tehtava}
	Esitä lauseke ilman itseisarvomerkkejä.
	\alakohdat{
		§ $2x-x|2-x|$
		§ $|x^2+3|$
		§ $|x^2-4|$
	}
	\begin{vastaus}
		\alakohdat{
			§ $\begin{cases}
					x^2, & \jos x \leq2 \\
					-x^2+4x, & \jos x>2 
				\end{cases}$
			§ $x^2+3$
			§ $\begin{cases}
					x^2-4, & \jos x \leq -2 \tai x \geq 2 \\
					-x^2+4, & \jos -2<x<2 
				\end{cases}$
		}
	\end{vastaus}
\end{tehtava}

\begin{tehtava}
	Esitä lauseke ilman itseisarvomerkkejä.
	\alakohdat{
		§ $(x-1)|4x^2+4x+1|$
		§ $|-3x^2+4x-2|$
	}
	\begin{vastaus}
		\alakohdat{
			§ $4x^3-3x-1$
			§ $3x^2-4x+2$
		}
	\end{vastaus}
\end{tehtava}

\begin{tehtava}
	Esitä lauseke ilman itseisarvomerkkejä.
	\alakohdat{
		§ $3|9-x^2|-2|x^2-9|$, kun $x\leq-3$
		§ $\dfrac{|-x^2+4x+5|}{|x^2-5x|}$, kun $x>5$
	}
	\begin{vastaus}
		\alakohdat{
			§ $x^2-9$ (vinkki: vastalukujen itseisarvot ovat yhtä suuret)
			§ $\frac{x+1}{x}$
		}
	\end{vastaus}
\end{tehtava}

\begin{tehtava}
	Ratkaise yhtälö.
	\alakohdat{
		§ $|x-2|=2$
		§ $|3x|=4$
		§ $|5x+7|=0$
		§ $|-4x+2|+1=0$
	}
	\begin{vastaus}
		\alakohdat{
			§ $x=0$ tai $x=4$
			§ $x=\frac{4}{3}$ tai $x=-\frac{4}{3}$
			§ $x=-\frac{7}{5}$
			§ ei ratkaisuja
		}
	\end{vastaus}
\end{tehtava}

\begin{tehtava}
	Ratkaise yhtälö.
	\alakohdat{
		§ $|x-2|=|3x|$
		§ $|3x|=|5x+7|$
		§ $|5x+6|=|5x+4|$
		§ $|-2x+2|=|x^2+2x+6|$
	}
	\begin{vastaus}
		\alakohdat{
			§ $x=-1$ tai $x=\frac{1}{2}$
			§ $x=-\frac{7}{2}$ tai $x=-\frac{7}{8}$
			§ $x=1$
			§ $x=-2$
		}
	\end{vastaus}
\end{tehtava}

\begin{tehtava}
	Ratkaise yhtälö.
	\alakohdat{
		§ $|-x| = |2x|$
		§ $|3x+1|= |7x-3|$
		§ $x^2+4|x|+4 = |x|^2+3|-x|+7$
		§ $|3x|+2 = |-2x|+1$
	}
	\begin{vastaus}
		\alakohdat{
			§ $x = 0$
			§ $x = 1$ tai $x = \frac{1}{5}$
			§ $x = \pm 3$
			§ Ei ratkaisuja.
		}
	\end{vastaus}
\end{tehtava}

\begin{tehtava}
	Ratkaise
	\alakohdat{
		§ $|x| = x^2$
		§ $3(|x|-1) = x^2-1$
		§ $2x^2+|x|-1 = 0$
	}
	\begin{vastaus}
		\alakohdat{
			§ $x = 0$ tai $x = \pm 1$
			§ $x = \pm 1$ tai $x = \pm 2$
			§ $x = \pm \frac{1}{2}$
		}
	\end{vastaus}
\end{tehtava}

\begin{tehtava}
	Maksimi $\max$ on kahden muuttujan funktio, joka liittää lukuihin niistä suuremman, siis
	\[
		\max(x,y) = \begin{cases}
		x, & \kun x \geq y \\
		y, & \kun x < y
		\end{cases}.
	\]
	Todista, että
	\[
		|x| = \max(x,-x)
	\]

	ja että
	\[
		\max(x,y) = \frac{x+y+|x-y|}{2}.
	\]
	\begin{vastaus}
		Käy eri tapaukset läpi.
	\end{vastaus}
\end{tehtava}

\begin{tehtava}
	Ratkaise yhtälö $|x+a| = |x|+|a|$ vapaan parametrin $a$ funktiona.
	\begin{vastaus}
		$x \in \rr$, jos $a=0$. $x>0$, jos $a>0$. $x<0$, jos $a<0$.
	\end{vastaus}
\end{tehtava}

\begin{tehtava}
	Itseisarvoyhtälöt voidaan ratkaista usein kätevästi neliöön korottamalla, vaikka yhtälön molemmat puolet eivät olisikaan välttämättä positiivisia. Tällöin on kuitenkin huomattava, että yhtälönratkaisun päättelyketjua ei voida suoraan suorittaa toiseen suuntaan: lopun ratkaisukandidaatit eivät välttämättä ole yhtälön ratkaisuja, mutta ne ovat ainoita mahdollisia, jolloin tarkistamalla ne yhtälö on ratkaistu. 

	Aina neliöminen ei kuitenkaan ole kannattavaa: neliöiminen saattaa johtaa identtisesti toteen yhtälöön/äärettömän moneen ratkaisukandidaattiin, joista ei välttämättä ole hyötyä alkuperäisen yhtälön ratkaisussa.

	Ratkaise yhtälöt.
	\alakohdat{
		§ $|x+1|-|2x-1| = x$
		§ $||2x-1|-|3x-2|| = x+1$
		§ $||2x-1|-|3x-2|| = x-1$
	}
	\begin{vastaus}
		\alakohdat{
			§ $x = 0$ tai $x = 1$
			§ $x = 0$
			§ $x \geq 1$
		}
	\end{vastaus}
\end{tehtava}

\end{tehtavasivu}
