\begin{tehtavasivu}

\subsubsection*{Opi perusteet}

\begin{tehtava}
	Määritä pisteiden etäisyys, kun pisteet ovat
	\alakohdat{
		§ $(1,2)$ ja $(1,5)$
		§ $(17,55)$ ja $(17,-1876)$
		§ $(-999999,423)$ ja $(999999,423)$.
	}
	\begin{vastaus}
		\alakohdat{
			§ $3$
			§ $1931$
			§ $1999998$
		}
	\end{vastaus}
\end{tehtava}

\begin{tehtava}
	Määritä pisteiden etäisyys, kun pisteet ovat
	\alakohdat{
		§ $(1,2)$ ja $(2,5)$
		§ $(8,5)$ ja $(17,0)$
		§ $(-12,34)$ ja $(56,-78)$.
	}
	\begin{vastaus}
		\alakohdat{
			§ $\sqrt{10}$
			§ $\sqrt{106}$
			§ $\sqrt{17168} = 4 \sqrt{1073}$
		}
	\end{vastaus}
	\end{tehtava}
	
\begin{tehtava}
	Laske kolmion $ABC$ piiri, kun sen kärjet ovat pisteet $A = (1,1)$, $B = (2,7)$ ja $C = (-4,-1)$.
	\begin{vastaus}
		$\sqrt{65}+10+\sqrt{29}$
	\end{vastaus}
\end{tehtava}

\begin{tehtava}
	Millä $t$:n arvoilla pisteiden $(1,t)$ ja $(-2t,3)$ etäisyys on
	\alakohdat{
		§ $\sqrt{10}$
		§ $4$
		§ $3t$
	}
	\begin{vastaus}
		\alakohdat{
			§ $t = 0$ tai $t = \frac{2}{5}$
			§ $t = \frac{1 \pm \sqrt{31}}{5}$
			§ $t = \frac{-1 \pm \sqrt{41}}{4}$
		}
	\end{vastaus}
\end{tehtava}

\subsubsection*{Hallitse kokonaisuus}

\subsubsection*{Sekalaisia tehtäviä}

TÄHÄN TEHTÄVIÄ SIJOITTAMISTA ODOTTAMAAN

\end{tehtavasivu}
