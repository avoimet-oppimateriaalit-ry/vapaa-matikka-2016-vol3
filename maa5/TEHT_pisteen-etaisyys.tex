\begin{tehtavasivu}

\subsubsection*{Opi perusteet}

\begin{tehtava} % tarkistettu / Niko
Määritä annetun pisteen etäisyys annetusta suorasta.
\alakohdat{
§ $(-1,-3),5x-2y+7 = 0 $
§ $(1,1),3x+5y-8 = 0 $
§ $(-7,17), x + 6 = 0$
§ $(e,\pi),\pi x+ e y = 0$
}
\begin{vastaus}
\alakohdat{
§ $\frac{8}{\sqrt{29}}$
§ $0$, eli piste on suoralla
§ $1$
§ $\frac{2\pi e}{\sqrt{\pi^2+e^2}}$
}
\end{vastaus}
\end{tehtava}

\begin{tehtava}
Määritä annetun pisteen etäisyys annetusta suorasta.
\alakohdat{
§ $(0,-3),y = x+3$
§ $(0,0),y = 2x+7$
§ $(-7,-7), y = 3$
§ $(-1,2), y = 5x-19$
}
\begin{vastaus}
\alakohdat{
§ $3\sqrt{2}$
§ $\frac{7\sqrt{5}}{5}$
§ $10$
§ $\sqrt{26}$
}
\end{vastaus}
\end{tehtava}

\begin{tehtava}
Kumpi annetuista pisteistä on lähempänä annettua suoraa?
\alakohdat{
§ Pisteet $(3,0)$ ja $(4,3)$, suora $5x-3y-4=0$
§ Pisteet $(2,4)$ ja $(3,3)$, suora $6x+7y-2=0$
§ Pisteet $(5,6)$ ja $(6,5)$, suora $x-y+1=0$
§ Pisteet $(3,7)$ ja $(3,0)$, suora $6x-y-15=0$
}
\begin{vastaus}
\alakohdat{
§ $(4,3)$
§ $(3,3)$
§ $(5,6)$
§ $(3,0)$
}
\end{vastaus}
\end{tehtava}

\begin{tehtava}
Kumpi annetuista pisteistä on lähempänä annettua suoraa?
\alakohdat{
§ Pisteet $(3,1)$ ja $(4,-2)$, suora $y=2x+1$
§ Pisteet $(-2,2)$ ja $(3,5)$, suora $y=5x+4$
§ Pisteet $(6,5)$ ja $(4,-1)$, suora $y=4x-10$
§ Pisteet $(-5,7)$ ja $(5,7)$, suora $y=2x+6$
}
\begin{vastaus}
\alakohdat{
§ $(3,1)$
§ $(-2,2)$
§ $(4,-1)$
§ $(5,7)$
}
\end{vastaus}
\end{tehtava}

\subsubsection*{Hallitse kokonaisuus}

\begin{tehtava}
Millä vakion $a$:n arvoilla pisteen $(a,1)$ etäisyys suorasta $x+2ay+a^2 = 0$ on $a$?

\begin{vastaus}
$a = \sqrt{\frac{11}{3}}+1$
\end{vastaus} 
\end{tehtava}

\subsubsection*{Sekalaisia tehtäviä}

\begin{tehtava}
Määritä kaikki pisteet $(x,y)$, jotka ovat yhtä etäällä suorista
\alakohdat{
§ $y = 0$ ja $y-2x = 0$
§ $y - 3x - 1 = 0$ ja $2x+4y -3= 0$
}
Pistejoukkoja kutsutaan \emph{kulmanpuolittajiksi}, miksi?
\begin{vastaus}
\alakohdat{
§ $(\sqrt{5}+1)x +2y = 0$ ja $(\sqrt{5}-1)x-2*y = 0$
§ $14(\sqrt{2}-1)x-14y+10+\sqrt{2} = 0$ ja $14(\sqrt{2}+1)x+14y-10+\sqrt{2} = 0$
Syntyneet suorat puolittavat alkuperäisten suorien määrämät kulmat.
}
\end{vastaus}
\end{tehtava}

\begin{tehtava}
\alakohdat{
§ Määritä kaikki ne suorat, joiden etäisyydet pisteistä $(0,1)$ ja $(2,-1)$ ovat yhtäsuuret.
§ Entä jos vaaditaan, että suora on yhtä etäällä myös pisteestä $(-1,-1)$
}
\begin{vastaus}
\alakohdat{
§ Suoria ovat pisteiden keskipisteen $(1,0)$ kautta kulkevat suorat, eli suorat muotoa $ax+by-a$, sekä pisteiden kautta kulkevan suoran kanssa yhdensuuntaiset suorat, eli suorat muotoa $y+x-c = 0$
§ Ne kolme suoraa, jotka kulkevat pisteiden määrittämän kolmion sivujen joidenkin kahden keskipisteen kautta, eli suorat $y = 0$, $2x-y-2 = 0$, sekä $2x+2y+1 = 0$
}
\end{vastaus}
\end{tehtava}

\begin{tehtava}
Määritä kaikki suorat, joiden etäisyys origosta on $1$
\begin{vastaus}
Suorat muotoa $ax+by \pm \sqrt{a^2+b^2} = 0$
\end{vastaus}
\end{tehtava}

\begin{tehtava}
  Määritä ne pisteet, joiden etäisyys suorasta $2x-y+4 = 0$ on $a$.
  \begin{vastaus}
    Pisteet $(x_0, y_0)$, jotka muodostavat suorat $2x_0 -y_0 +4 \mp \sqrt{5}a = 0$.
  \end{vastaus}
\end{tehtava}


\begin{tehtava}
Matti haluaa uimaan pitkälle suoralle joelle. Hän tietää, että jos hän matkaa suoraan pohjoiseen, matkaa kertyy $10$ kilometriä, mutta jos hän päättää lähteä suoraan itään, hänen on käveltävä vain $7,0$ kilometriä. Kuinka kaukana Matti on joesta?
\begin{vastaus}
$5,7$ kilometrin etäisyydellä
\end{vastaus}
\end{tehtava}

\begin{tehtava}
Matti on (autiolla) kolmion muotoisella saarella. Saaren eteläkärki on $5.0$ kilometrin päässä Matista, suoraan etelään, toinen saaren kärjistä löytyy $8.0$ kilmetrin etäisyydeltä suoraan idästä, ja kolmas suoraan luoteesta, $11$ kilometrin etäisyydeltä. Jos Matti haluaa lähimmälle rannalle, mille rannoista hänen on suunnattava, ja kuinka paljon matkaa kertyy?
\begin{vastaus}
Lähin ranta on etelä- ja luoteiskärkiä yhdistävä, ja matkaa sille kertyy n. $2.6$ kilometriä
\end{vastaus}
\end{tehtava}

\begin{tehtava}
  Todista kaava pisteen pisteen $P=(x_0,y_0)$ etäisyydelle yleisestä suorasta $l$, jonka yhtälö on $ax + by +c = 0$ yleistämällä yhdenmuotoisten kolmioiden menetelmää.
  \alakohdat{
      § Voit olettaa ensin, että (esimerkin kuvan merkinnöin) koordinaattiakselien ja suoran $l$ väliin jäävä kolmio $OAB$ on olemassa.
      § Entä kun kolmiota $OAB$ ei ole olemassa? Milloin näin tapahtuu?
  }
  \begin{vastaus}
    \alakohdat{
        § \emph{Vihje.} Etsi nyt suoran $l$ yhtälön avulla yhtälöt pisteiden $A, B$ ja $R$ koordinaateille. $A, B:$ Origo on edelleen $(0,0)$. $R:$ $R = (x_r, y_r) = (x_0, y_r)$, ratkaise $y_r$.
        § \emph{Vihje.} Janan $AB$ pituus.
    }
  \end{vastaus}
\end{tehtava}

\begin{tehtava}
	Pisteen etäisyyden suorasta voi laskea myös kurssi Geometria (MAA3) tietojen avulla. Jos pisteet $A$ ja $B$ ovat suoralla $l$, kolmion $ABP$ ala voidaan laskea kolmion sivujen pituuksien, mutta myös sivun $AB$ pituuden, ja sitä vastaan piirretyn korkeusjanan pituuden avulla. Kun vielä huomataan, että korkeusjanan pituus on yhtäsuuri kuin pisteen $P$ etäisyys suorasta $l$. Heronin kaava antaa yhteyden kolmion pinta-alan ja sen sivujen pituuksien välille:
	\[
	A = \sqrt{p(p-a)(p-b)(p-c)},
	\]
	missä $a$, $b$ ja $c$ ovat kolmion sivujen pituudet, ja $p = (a+b+c)/2$.
	
	Määritä origon etäisyys suorasta, jolla on pisteet $(-1,1)$ ja $(3,4)$.
	\begin{vastaus}
		$\frac{7}{5}$
	\end{vastaus}
\end{tehtava}


TÄHÄN TEHTÄVIÄ SIJOITTAMISTA ODOTTAMAAN

\end{tehtavasivu}
