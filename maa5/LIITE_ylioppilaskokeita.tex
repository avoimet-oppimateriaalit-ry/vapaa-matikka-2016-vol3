\section{Tehtäviä ylioppilaskokeista}

%%Pitäisikö tämän olla TEHT_ylioppilaskokeet kuten MAA1 ja MAA2? Nyt esim vastaukset ei toimi.
%%Vastaukset ei mene kohilleen ja nämä on vastausten jälkeen
%%Eikä vastauksia ole

\subsubsection*{Lyhyen oppimäärän tehtäviä}


\begin{tehtava} (K2014/2 a ja b)
\alakohdat{
		§ Missä pisteessä suora $x-5y=4$ leikkaa $y$-akselin?
		§ Ratkaise yhtälö $4x^3=48$. Anna tarkka arvo ja kolmidesimaalinen likiarvo.
	}
\end{tehtava}
%Lisännyt Aleksi Sipola 17.5.2014

\begin{tehtava}  (S2013/2 a ja b)
\alakohdat{
		§ Missä pisteissä suora $y=-3x+12$ leikkaa koordinaattiakselit?
		§ Ratkaise yhtälöpari 
		\[
\left\{
\begin{aligned}
 2x+y=4  \\
 -x+2y=1  
\end{aligned}
\right. 
\]
	}
\end{tehtava}
%Lisännyt Aleksi Sipola 17.5.2014

\begin{tehtava} (S2013/5) TARVITSEE KUVAN
Oheinen kuvaaja esittää paraabelia $y=ax^2+bx+c$. Määritä vakiot $a$, $b$ ja $c$ käyttämällä kuvioon ympyröillä merkittyjä pisteitä.
\end{tehtava}
%Lisännyt Aleksi Sipola 17.5.2014

\begin{tehtava} (K2013/2)
\alakohdat{
		§ Millä muuttujan $x$ arvoilla $4x+17$ on suurempi kuin $2-x$?
		§ Ratkaise yhtälö $x^2+14=-49$.
		§ Suora kulkee origon ja pisteen $(2,3)$ kautta. Kulkeeko se myös pisteen $(48,75)$ kautta?
	}
\end{tehtava}
%Lisännyt Aleksi Sipola 17.5.2014

\begin{tehtava} (K2012/4)
\alakohdat{
		§ Funktion $f(x)=\frac{3}{2}x+b$ nollakohta $2$. Määritä vakion $b$ arvo
		§ Ratkaise yhtälö $x^2+14=-49$.
		§ Suora kulkee origon ja pisteen $(2,3)$ kautta. Kulkeeko se myös pisteen $(48,75)$ kautta?
	}
\end{tehtava}
%Lisännyt Aleksi Sipola 17.5.2014


\begin{tehtava} (S2012/1)
\alakohdat{
		§ Ratkaise yhtälö $x^2-2x=0$.
		§ Ratkaise yhtälö $\frac{2}{3}x-1=\frac{2}{3}$.
		§ Ratkaise yhtälöpari
		\[
\left\{
\begin{aligned}
 x+2y=-4  \\
 2x-y=-3  
\end{aligned}
\right. 
\]
	}
\end{tehtava}
%Lisännyt Aleksi Sipola 17.5.2014

\begin{tehtava} (S2012/11) 
Aikuisen ihmisen sääriluun pituus $y$ riippuu henkilön pituudesta $x$ kaavojen 
\[
\begin{aligned}
 y=0,43-27 \text{(nainen)} \\
 y=0,45x-31  \text{(mies)}
\end{aligned}
\]
mukaisesti,kun yksikkönä on senttimetri.

  \alakohdat{
		§ Arkeologi löytää naisen sääriluun, joka on 41 cm pitkä. Kuinka pitkä nainen oli?
		§ Kaivauksissa löytyneen miehen pituudeksi arvioidaan 175 cm. Miehen läheltä löytyy sääriluu, jonka pituus on 42 cm. Onko  kyseessä saman henkilön sääriluu?		
  }
 

\end{tehtava}
%Lisännyt Aleksi Sipola 17.5.2014



\begin{tehtava} (K2011/1)
\alakohdat{
		§ Ratkaise yhtälö $4x+(5x-4)=12+3x$.
		§ Sievennä lauseke $x^2+x-(x^2-x)$ ja laske sen arvo, kun $x=\frac{1}{2}$
		§ Ratkaise yhtälöpari
		\[
\left\{
\begin{aligned}
 x-2y=0  \\
 x-3y=1  
\end{aligned}
\right. 
\]
	}
\end{tehtava}
%Lisännyt Aleksi Sipola 17.5.2014


\begin{tehtava}  (S2011/4)
Ludwig van Beethoven, Wolfgang Amadeus Mozart ja Johann Sebastian Bach elivät yhteensä 156 vuotta. Bach eli yhdeksän vuotta vanhemmaksi kuin Beethoven, Mozart kuoli 21 vuotta nuorempana kuin Beethoven. Kuinka vanhoiksi säveltäjät elivät?
\end{tehtava}
%Lisännyt Aleksi Sipola 17.5.2014

\begin{tehtava}  (S2010/1b)
Ratkaise yhtälö $(x-2)^2-4(2-x)=0$
\end{tehtava}
%Lisännyt Aleksi Sipola 17.5.2014

\begin{tehtava}  (S2010/3)
Oheisessa kuviossa on kaksi suoraa. Määritä näiden yhtälöt, ja laske niiden leikkauspisteen koordinaatit. Mikä on suorien ja $y$-akselin raajaman kolmion pinta-ala?
\end{tehtava}
% \begin{kuva}
%     kuvaaja.pohja(-2, 4, -2, 4, korkeus = 6, nimiX = "$x$", nimiY = "$y$", ruudukko = True)
%     kuvaaja.piirra("-1.5*x+3")
%     kuvaaja.piirra("1", kohta = 1, suunta = 0)
% \end{kuva}
%Lisännyt Aleksi Sipola 17.5.2014


\begin{tehtava}  (S2010/13)
Millä vakion $a$ arvolla yhtälöparilla 
\[
\left\{
\begin{aligned}
 2x+(a+1)y=5  \\
 3x+(a-2)y=a  
\end{aligned}
\right. 
\]
ei ole ratkaisua?
\end{tehtava}
%Lisännyt Aleksi Sipola 17.5.2014

\begin{tehtava}  (K2010/7)
Suorakulmaisen kolmion kateettien pituudet ovat $3,2 cm$ ja $5,7 cm$. Laske hypotenuusan pituus ja suoran kulman kärjen etäisyys hypotenuusasta.
\end{tehtava}
%Lisännyt Aleksi Sipola 17.5.2014

\begin{tehtava}  (K2010/10)
Määritä sen suoran yhtälö, joka kulkee pisteiden $A=(-1,1)$ ja $B=(8,4)$ yhdysjanan keskipisteen kautta ja on kohtisuorassa tätä janaa vastaan. Missä pisteissä suora leikkaa koordinaattiakselit? Piirrä kuvio.
\end{tehtava}
%Lisännyt Aleksi Sipola 17.5.2014

\begin{tehtava}  (K2010/13)
Määritä sen suoran yhtälö, joka kulkee pisteiden $A=(-1,1)$ ja $B=(8,4)$ yhdysjanan keskipisteen kautta ja on kohtisuorassa tätä janaa vastaan. Missä pisteissä suora leikkaa koordinaattiakselit? Piirrä kuvio.
\end{tehtava}
%Lisännyt Aleksi Sipola 17.5.2014


\begin{tehtava}  (S2009/12)
Suorat $x+y=8, x+3y=18$ ja $y-3=0$ rajoittavat kolmion. Piirrä kuvio ja laske kolmion pinta-ala. Muodosta epäyhtälöryhmä, jonka ratkaisuna on piirtämäsi kolmio sivut mukaan lukien.
\end{tehtava}
%Lisännyt Aleksi Sipola 17.5.2014

\begin{tehtava}  (S2009/3)
\alakohdat{
		§ Suoran kulmakerroin on $-\frac{1}{3}$, ja suora kulkee pisteen $(-1,2)$ kautta. Esitä suoran yhtälö muodossa $y=kx+b$
		§ Tutki, millä muuttujan $x$ arvoilla polynomi $2x^2+5x-3$ saa negatiivisia arvoja.
	}
\end{tehtava}
%Lisännyt Aleksi Sipola 17.5.2014


\begin{tehtava}  (S2008/8)
Millä vakion $a$ arvoilla suorat $y=-3x+2$ ja $y=ax+6$ erottavat $x$-akselista janan, jonka pituus on 3?
\end{tehtava}
%Lisännyt Aleksi Sipola 17.5.2014


\begin{tehtava}  (S1959/4)
Määrää $a$ siten, että paraabelin $y=ax^2$ suorasta $y=x+1$ erottama jänne on 8 pituudenyksikköä. Piirrä kuvio.
\end{tehtava}
%Lisännyt Aleksi Sipola 17.5.2014

\begin{tehtava}(S1959/9)
Kolme ympyrää, joiden säteet ovat $1cm$, $2cm$ ja $3cm$, sivuaa toisiaan ulkopuolisesti. Laske niiden kaarien rajoittaman (pienimmän) ``kaarikolmion'' piiri. 
\end{tehtava}
%Lisännyt Aleksi Sipola 17.5.2014

\begin{tehtava}(S1959/10)
Millä $k$:n reaaliarvoilla yhtälön $kx^2+2kx+3x+2k+1=0$ juuret ovat samanmerkkisiä reaalilukuja? 
\end{tehtava}
%Lisännyt Aleksi Sipola 17.5.2014

\subsubsection*{Pitkän oppimäärän tehtäviä}

\begin{tehtava}(S07/1b)
	Muodosta sen suoran yhtälö, joka kulkee pisteiden $(4, -3)$ ja $(-2,6)$ kautta. 
\end{tehtava}

\begin{tehtava}(K06/1b)
	Missä pisteessä suora $y=3x-4$ leikkaa x-akselia? 
\end{tehtava}

\begin{tehtava} (S07/5)
	Määritä ympyrän $x^2+y^2+4x-2y+1=0$ niiden tangettien yhtälöt, jotka kulkevat pisteen $(1,3)$ kautta.
\end{tehtava}



\begin{tehtava}(K2014/5)
Ympyrä sivuaa suoraa $3x-4y=0$ pisteessä $(8,6)$. Lisäksi se sivuaa positiivista $x$-akselia.
Määritä ympyrän keskipiste ja säde. 
\end{tehtava}
%Lisännyt Aleksi Sipola 17.5.2014
% VASTAUKSET EIVÄT MENE VASTAUS OSIOON VAAN JÄÄVÄT TEHTÄVÄNANNON PERÄÄN 
 % \begin{vastaus} VASTAUKSET EIVÄT MENE VASTAUS OSIOON VAAN JÄÄVÄT TEHTÄVÄNANNON PERÄÄN 
%	Ympyrän keskipiste on $(10,frac[10][3])$ ja $r=30$
 %   \end{vastaus}


\begin{tehtava}(K2014/9)
Taso $x+2y+3z=6$ leikkaa positiiviset koordinaattiakselit pisteissä $A$, $B$ ja $C$.
\alakohdat{
		§ Määritä sen tetraedrin tilavuus, jonka kärjet ovat origossa $O$ sekä pisteissä $A$,$B$ ja $C$. 
		§ Määritä kolmion $ABC$ pinta-ala
	}
% 	VASTAUKSET EIVÄT MENE VASTAUS OSIOON VAAN JÄÄVÄT TEHTÄVÄNANNON PERÄÄN 
%  \begin{vastaus} 
%	 Tilavuus on $6$
%  \end{vastaus}
\end{tehtava}
%Lisännyt Aleksi Sipola 17.5.2014

\begin{tehtava} (S2013/1)

\alakohdat{
		§ Ratkaise yhtälöä $x^2+6x=2x^2+9$.
		§ Ratkaise yhtälö $\frac{1+x}{1-x}=\frac{1-x^2}{1+x^2}$
		§ Esitä polynomi $x^2-9x+14$
	}
\end{tehtava}
%Lisännyt Aleksi Sipola 17.5.2014

\begin{tehtava}(S2013/10)
Pöydällä on kolme samankokoista palloa, joista kukin koskettaa kahta muuta. Niiden päälle asetetaan neljäs samanlainen  pallo, joka koskettaa kaikkia kolmea alkuperäistä palloa.
Mikä on rakennelman korkeus? Anna vastauksena tarkka arvo pallojen säteen avulla lausuttuna. 
\end{tehtava}
%Lisännyt Aleksi Sipola 17.5.2014

\begin{tehtava} (S2013/*14)
Tarkastellaan tasokäyrää, jonka yhtälö on $2x^2+2y^2-3xy-2x+2y-4=0$.
\alakohdat{
		§ Määritä käyrän ja koordinaattiakselien leikkauspisteet. (2p.)
		§ Osoita, että kaikki leikkauspisteet ovat saman ympyrän kehällä, ja määritä tämän ympyrän yhtälö. (3p.)
		§ Suora kulkee origon ja b-kohdan ympyrän keskipisteen kautta. Missä pisteissä tämä suora leikkaa alkuperäisen käyrän? (2p.)
		§ Onko alkuperäinen käyrä ympyrä? (2p.)
	}
\end{tehtava}
%Lisännyt Aleksi Sipola 17.5.2014

\begin{tehtava} (K2013/1)
\alakohdat{
		§ Ratkaise yhtälöä $(x-4)^2=(x-4)(x+4)$.
		§ Ratkaise epäyhtälöä $\frac{3}{5}x-\frac{7}{10} < -\frac{2}{15}x$.
		§ Suora kulkee pisteiden $(1,7)$ ja $2,4$ kautta. Missä pisteessä se leikkaa $x$-akselin
	}
\end{tehtava}
%Lisännyt Aleksi Sipola 17.5.2014


\begin{tehtava}(K2013/10) KUVA TARVITAAN
Oheisen kuution särmän pituus on 2. Sen sisällä on vaaleanpunainen pallo,joka sivuaa jokaista kuution tahkoa. Kuution yhdessä kulmassa on pienempi sininen pallo, joka sivuaa suurta palloaja kolmea kuution tahkoa kuvion mukaisesti. Laske sinisen pallon säteen tarkka arvo. 
\end{tehtava}
%Lisännyt Aleksi Sipola 17.5.2014


\begin{tehtava}(K2013/*15a) KUVA TARVITAAN
Ympyrä, jonka säde on $r>\frac{1}{2}$, asetetaan paraabelin $y=x^2$ sisäpuolelle alla olevan kuvan mukaisesti. Näytä, että ympyrän keskipisteen $y$‐koordinaatti on $r^2+\frac{1}{4}. $ (3
p.)
\end{tehtava}
%Lisännyt Aleksi Sipola 17.5.2014

\begin{tehtava} (S2012/1) Ratkaise yhtälöt
  \alakohdat{
		§ $2(1-3x+3x^2)=3(1+2x+2x^2)$
		§ $|x|=1+x$
		§ $1-x=\frac{1}{1-x}$	
  }
\end{tehtava}
%Lisännyt Aleksi Sipola 17.5.2014

\begin{tehtava}(S2012/*15)
Suora ympyrälieriö on pallon sisällä niin, että sen molempien pohjien reunat sivuavat pallon pintaa. Pallon pinta‐alan suhdetta lieriön koko pinta‐alaan merkitäänsymbolilla. $t$ Lieriön kokopinta‐alalla tarkoitetaan sen vaipan ja pohjien yhteenlaskettuja pinta‐aloja. 
\alakohdat{
		§ Määritä lieriön korkeuden suhde lieriön pohjan säteeseen parametrin $t$ avulla lausuttuna. (2p.) \\
		\\
		Millä parametrin $t$ arvoilla
		§ tällaista lieriötä ei ole olemassa (2p.)
		§ on täsmälleen yksi tällainen lieriö (3p.)	
		§ on kaksi tällaista lieriötä? (2p.)	
  }
\end{tehtava}

\begin{tehtava} (K2012/*15) Ratkaise   KUVA TARVITAAN
\alakohdat{
		§ Kaksi ympyrää sivuaa toisiaan ja $x$-akselia kuvan 1 mukaisesti. Määritä ympyröiden keskipisteiden vaakasuora etäisyys $d$ niiden säteiden avulla lausuttuna. (3 p.)
		§ Kolme ympyrää sivuaa toisiaan ja $x$-akselia kuvan 2 mukaisesti. Määritä keskimmäisen ympyrän säde $r_3$ kahden reunimmaisen ympyrän säteiden avulla lausuttuna. (3 p.)
		§ Todista René Descartesin (1596-1650) keksimä b-kohdan ympyröihin liittyvä kaava
		
		$(k_1+k_2+k_3)^2=2(k_1^2+k_2^2+k_3^2)$, \\
		jossa $k_i=\frac{1}{r_i}$,  $i=1,2,3.$ (3 p.)
	}
\end{tehtava}
%Lisännyt Aleksi Sipola 17.5.2014


\begin{tehtava} (S2011/1b) Ratkaise
Laske suoran $y=2x$ ja ympyrän $x^2+y^2=1$ leikkauspisteet.
\end{tehtava}
%Lisännyt Aleksi Sipola 17.5.2014

\begin{tehtava} (K2011/1) Ratkaise
  \alakohdat{
		§ $\frac{2}{x}=\frac{3}{x-2}$
		§ $x^2-2\leq x$
		§ $\left|\frac{3}{2}x-6\right|=6$	
  }
\end{tehtava}
%Lisännyt Aleksi Sipola 17.5.2014

\begin{tehtava}(K2010/4)
Puolipallon sisällä on kuutio siten, että sen yksi sivutahko on puolipallon pohjatasolla ja vastakkaisen sivutahkon kärkipisteet ovat pallopinnalla. Kuinka monta prosenttia kuution tilavuus on puolipallon tilavuudesta?
\end{tehtava}

\begin{tehtava}(K2010/8)
Tietunnelin poikkileikkaus on osa alaspäin aukeavaa paraabelia. Tien leveys on $10 m$, ja tunnelin poikkileikkauksen pinta-ala on $25 m^2$. Määritä tunnelin korkeus senttimetrin tarkkuudella.
\end{tehtava}


\begin{tehtava}(S2009/9)
Mikä paraabelin $y=5-x^2$ piste on lähinnä origoa? Piirrä kuvio.
\end{tehtava}

\begin{tehtava}(K2009/1 a ja b)
  \alakohdat{
		§ Ratkaise epäyhtälö $(x-3)^2=(x-1)(x+1)$
		§ Määritä suorien $\frac{x}{3}+\frac{y}{2}=1$ ja $3x-2y+3=0$ leikkauspiste	
  }
\end{tehtava}

\begin{tehtava} (K2008/2 a ja c)
  \alakohdat{
		§ Määritä suorien $2x+y=8$ ja $3x+2y=5$ leikkauspiste
		§ Ratkaise yhtälö $|3x-2|=5$	
  }
\end{tehtava}
%Lisännyt Aleksi Sipola 17.5.2014


\begin{tehtava}(K2008/7a)
Laske paraabelien $y=x^2-3$ ja $y=-x^2+2x+1$ leikkauspisteiden koordinaatit.
\end{tehtava}
%Lisännyt Aleksi Sipola 17.5.2014

\begin{tehtava}(K1983/7)
Ympyrä ja piste sen ulkopuolella ovat tunnetut. Hae pistettä ja ympyrän keskipistettä yhdistävällä suoralla viivalla semmoinen piste jonka etäisyys tunnetusta pisteestä on yhtäsuuri kuin ne tangentit, jotka haettavasta pisteestä saatetaan piirtää ympyrälle. 
\end{tehtava}
%Lisännyt Aleksi Sipola 17.5.2014


\begin{tehtava}(K1941/3)
Määrää $a$ siten, että yhtälön $x^2+(a+2)x-a^2=0$ suuremman ja pienemmän juuren eroitus saa mahdollisimman pienen arvon. Mitkä ovat juuret? 
\end{tehtava}
%Lisännyt Aleksi Sipola 17.5.2014


\begin{tehtava}(S1893/3)
Ratkaise ekvationit:
\[
\left\{
\begin{aligned}
 x+\frac{1}{2}(y+z)=102    \\
 y+\frac{1}{2}(x+z)=78  \\
 z+\frac{1}{2}(x+y)=61
\end{aligned}
\right. 
\]
\end{tehtava}
%Lisännyt Aleksi Sipola 17.5.2014

