\begin{tehtavasivu}

\paragraph*{Opi perusteet}

\begin{tehtava}
	Ympyrän keskipiste on $(0, 0)$ ja säde $5$. Muodosta ympyrän yhtälö.
	\begin{vastaus}
		$x^2+y^2=25$
	\end{vastaus}
\end{tehtava}

\begin{tehtava}
	Ympyrän keskipiste on $(1, -4)$ ja säde $10$. Muodosta ympyrän yhtälö.
	\begin{vastaus}
		$(x-1)^2+(y+4)^2=100$ tai auki kirjoitettuna $x^2+y^2-2x+8y-83=0$ % auki kirjoitettu vai aukikirjoitettu
	\end{vastaus}
\end{tehtava}

\begin{tehtava}
	Määritä keskipiste ja säde.
	\alakohdat{
		§ $(x-3)^2+(y+7)^2=12$
		§ $x^2+y^2=49$
	}
	\begin{vastaus}
		\alakohdat{
			§ keskipiste $(3, -7)$, säde $2\sqrt{3}$
			§ keskipiste $(0, 0)$, säde $7$
		}
	\end{vastaus}
\end{tehtava}

\begin{tehtava}
	Määritä keskipiste ja säde.
	\alakohdat{
		§ $x^2+y^2-10x+16y+72=0$
		§ $x^2+y^2+8x-22y+129=0$
	}
	\begin{vastaus}
		\alakohdat{
			§ keskipiste $(5, -8)$, säde $\sqrt{17}$
			§ keskipiste $(-4, 11)$, säde $2\sqrt{2}$
		}
	\end{vastaus}
\end{tehtava}

\begin{tehtava}
	Määritä ympyrän $(x+10)^2+y^2=2$ keskipiste ja säde ja ratkaise ympyrän yhtälöstä $y$. 
	\begin{vastaus}
		keskipiste $(-10, 0)$, säde $\sqrt{2}$, $y=\pm\sqrt{2-(x+10)^2}$ 
	\end{vastaus}
\end{tehtava}

\begin{tehtava}
	Ympyrän keskipiste on origo ja säde $3$. Mitkä seuraavista pisteistä ovat ympyrän kehällä?
	\alakohdat{
		§ $(10, -2)$
		§ $(-3, 0)$
		§ $(2, \sqrt{5})$
		§ $(1, 3)$
	}
	\begin{vastaus}
		b) ja c)
	\end{vastaus}
\end{tehtava}

\begin{tehtava}
	Määritä $k$ niin, että lauseke $(x-3)^2+(y+3)^2=k$ on
	\alakohdat{
			§ ympyrä
			§ $\sqrt{7}$-säteinen ympyrä
			§ origon kautta kulkeva ympyrä?
	}
	\begin{vastaus}
		\alakohdat{
			§ $k>0$
			§ $k=7$
			§ $k=18$
		}
	\end{vastaus}
\end{tehtava}

\begin{tehtava}
	Ympyrän keskipiste on $(3,5)$ ja piste $(4,8)$ on ympyrällä. Määritä ympyrän yhtälö.
	\begin{vastaus}
		$(x-3)^2+(y-5)^2 = 10$
	\end{vastaus}
\end{tehtava}

\begin{tehtava}
	Tutki, mitä yhtälöiden kuvaajat esittävät.
	\alakohdat{
		§ $x^2+y^2-6x+4y+4=0$
		§ $x^2+y^2+14x-6y+10=0$
	}
	\begin{vastaus}
		\alakohdat{
			§ ympyrä
			§ piste
		}
	\end{vastaus}
\end{tehtava}

\paragraph*{Hallitse kokonaisuus}

\begin{tehtava}
	Määritä ympyrän keskipiste ja säde.
	\alakohdat{
		§ $(x+t)^2+(y+u)^2=k, k>0$
		§ $(x+2)^2+(y-7)^2=-8$
	}
	\begin{vastaus}
		\alakohdat{
			§ keskipiste $(-t, -u)$, säde  $\sqrt{k}$
			§ ei ole ympyrä
		}
	\end{vastaus}
\end{tehtava}

\begin{tehtava}
	Ympyrä sivuaa $y$-akselia pisteessä $(0, -1)$ ja kulkee pisteen $(3, 2)$ kautta. Mikä on ympyrän yhtälö?
	\begin{vastaus}
		$(x-3)^2+(y+1)^2=9$
	\end{vastaus}
\end{tehtava}

\begin{tehtava}
	Kolmen tunnetun pisteen kautta kulkevan ympyrän yhtälö voidaan määrittää monella eri tavalla. Ympyrä kulkee pisteiden $(1, 6), (-2, 5)$ ja $(5, 4)$ kautta. Määritä ympyrän yhtälö seuraavilla tavoilla:
	\alakohdat{
		§ Tutki, millä vakioiden $x_0$, $y_0$ ja $r$ arvoilla pisteet ovat ympyrällä.
		§ Ympyrän keskipiste on minkä tahansa kahden pisteen keskinormaalilla, joten sen voi määrittää kahden eri keskinormaalin leikkauspisteenä.
	}
	\begin{vastaus}
		$(x-1)^2+(y-1)^2=16$
	\end{vastaus}
\end{tehtava}

\begin{tehtava}
	Jana, jonka pituus on $t$ liikkuu koordinaatistossa siten, että sen toinen pää on $x$-akselilla ja toinen $y$-akselilla. Mitä käyrää pitkin liikkuu janan keskipiste?
	\begin{vastaus}
		$x^2+y^2=\frac{1}{4}t^2$
	\end{vastaus}
\end{tehtava}

\begin{tehtava}
	Millä $c$:n reaaliarvoilla yhtälö $x^2-2xc+y^2-2c-2 = 0$ esittää ympyrää? Mikä on tällöin ympyrän keskipiste ja säde? Todista, että ympyrä kulkee tällöin kahden $c$:stä riippumattoman pisteen kautta.
	\begin{vastaus}
		Kaikilla, keskipiste $(c,0)$, säde $\sqrt{(c-1)^2+1}$. Ympyrät kulkevat aina pisteiden $(1,\pm 1)$ kautta.
	\end{vastaus}
\end{tehtava}

\begin{tehtava}
	Ympyrä voidaan määritellä myös monella muulla yhtäpitävällä tavalla
	\alakohdat{
		§ Jos $A = (1,0)$ ja $B = (-1,0)$, määritä kaikki ne pisteet $P$, joilla $AP$ ja $BP$ ovat kohtisuorassa.
		§ Jos $A = (2,0)$ ja $B = (-1,0)$, määritä kaikki ne pisteet $P$, joille $\frac{AP}{BP} = 2$.
		§ Jos $A = (3,0)$ ja $B = (-1,0)$, määritä kaikki ne pisteet $P$, joille $AP^2+BP^2 = 10$.
	}
	\begin{vastaus}
		\alakohdat{
			§ Ympyrän $x^2+y^2 = 1$ pisteet (lukuunottamatta pisteitä $A$ ja $B$)
			§ Ympyrän $(x+2)^2+y^2 = 4$ pisteet
			§ Ympyrän $(x-1)^2+y^2 = 1$ pisteet
		}
	\end{vastaus}
\end{tehtava}

\paragraph*{Sekalaisia tehtäviä}

TÄHÄN TEHTÄVIÄ SIJOITTAMISTA ODOTTAMAAN

\end{tehtavasivu}
