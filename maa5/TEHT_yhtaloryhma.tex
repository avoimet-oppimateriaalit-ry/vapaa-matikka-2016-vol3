\begin{tehtavasivu}

\subsubsection*{Opi perusteet}

\begin{tehtava}
  Mitkä seuraavista ovat lineaarisia yhtälöitä?
  \alakohdat{
    § $2x^2 + y +1 = 0$
    § $y = 0$
    § $3x - 2y = 5$
  }
  \begin{vastaus}
    \alakohdat{
      § Ei.
      § On.
      § On.
    }
  \end{vastaus}
\end{tehtava}

\begin{tehtava}
    Ratkaise yhtälöparit.
    \begin{align*}
        x-2y &= 0 \\
        -2x+y+3 &=0
    \end{align*}
    \begin{vastaus}
        $x = 2, \, y = 1$
    \end{vastaus}
\end{tehtava}

\begin{tehtava}
    Ratkaise yhtälöpari.
    \begin{align*}
        x+y+1 &= 0 \\
        x+2y+1 &=0
    \end{align*}
    \begin{vastaus}
        $x = -1, \, y = 0$
    \end{vastaus}
\end{tehtava}

\begin{tehtava}
    Ratkaise yhtälöpari.
    \begin{align*}
        2x+5y+1 &= 0 \\
        2x+2y+7 &=0
    \end{align*}
    \begin{vastaus}
        $x = -\frac{11}{2}, \, y = 2$
    \end{vastaus}
\end{tehtava}

\begin{tehtava}
    Ratkaise yhtälöpari.
    \begin{align*}
        2a+3b &= 8 \\
        6a+2b &= -4
    \end{align*}
    \begin{vastaus}
        $a = -2, \, b = 4$
    \end{vastaus}
\end{tehtava}

\begin{tehtava}
    Ratkaise yhtälöpari.
    \begin{align*}
        \frac{x}{3}+\frac{y}{7} + 1 &= 3 \\
        x - \frac{y-1}{3} &= -5
    \end{align*}
    \begin{vastaus}
        $x = -45, \, y = 119$
    \end{vastaus}
\end{tehtava}

\begin{tehtava}
	Ratkaise yhtälöpari.
	\begin{align*}
		x^2-y+1 &= 0 \\
		x+y-2 &= 0
	\end{align*}
	\begin{vastaus}
		$x = \frac{-1+\sqrt{5}}{2}, y = \frac{10-2\sqrt{5}}{4}$ tai $x = \frac{-1-\sqrt{5}}{2}, y = \frac{10+2\sqrt{5}}{4}$
	\end{vastaus}
\end{tehtava}

\begin{tehtava}
    Ratkaise yhtälöpari. $t \in \rr$ on vapaa parametri, joka saa sisältyä vastaukseen.
    \begin{align*}
        x+2y-t-1 &= 0 \\
        x+y+t^2 &=0
    \end{align*}
    \begin{vastaus}
        $x = -2t^2-t-1, \, y = t^2+t+1$
    \end{vastaus}
\end{tehtava}

\begin{tehtava}
    Ratkaise yhtälöryhmä.   
    \begin{align*}
        x+2y+1 &=0 \\
        x+2z+3 &=0 \\
        y+2z+5 &=0
    \end{align*}
    \begin{vastaus}
        $x = 1, \, y = -1, \, z = -2$
    \end{vastaus}
\end{tehtava}

\subsubsection*{Hallitse kokonaisuus}

\begin{tehtava}
  Esimerkeissä huomasimme, että lineaarisella yhtälöparilla on 0, 1 tai äärettömän monta ratkaisua ja tällä havainnolla oli luonteva geometrinen tulkinta.
  Perustellaan esittämäämme lausetta lineaarisen yhtälöryhmän muuttujien ja yhtälöiden määrän suhteesta ratkaisujen lukumäärään tutkimalla kahden muuttujan yhtälöryhmän ratkaisujen lukumäärää, kun yhtälöitä on enemmän.
  \alakohdat{
    § Tutki (esimerkiksi piirtämällä) montako ratkaisua seuraavalla yhtälöryhmällä on:
      \[
        \left\{
          \begin{aligned}
            a_1x+b_1y+c_1 &= 0 \\
            a_2x+b_2y+c_2 &= 0 \\
            a_3x+b_3y+c_3 &= 0
          \end{aligned}
        \right.
      \]
      \emph{Vihje.} Monellako tavalla kolme suoraa voi leikata toisensa tasossa?
    § Entä jos yhtälöryhmään lisätään neljäs tai useampi suora?
    § Milloin kahden muuttujan lineaarisessa yhtälöryhmässä on enemmän muuttujia kuin yhtälöitä? Montako ratkaisua tällöin on? 
  }
  \begin{vastaus}
    \alakohdat{
      § Kolme tason suoraa voivat olla keskenään yhdensuuntaisia leikkaamatta toisiaan (0 ratkaisua), leikata kaikki toisensa samassa pisteessä (1 ratkaisu) tai olla saman suoran yhtälöitä(äärettömän monta ratkaisua). Suorat voivat myös leikata toisiaan siten, että kolmesta suorasta kaksi leikkaavat toisensa jossain pisteessä, jonka kautta kolmas suora ei kuitenkaan kulje. Tällöin ratkaisuja on myös 0 (piirrä kuva.)

        Ratkaisuja on 0, 1 tai äärettömän paljon.
      § Sama kuin yllä pätee kun suoria lisätään.
      § 'Yhtälöryhmässä' on tällöin vain yksi suoran yhtälö.
    }
  \end{vastaus}
\end{tehtava}

\begin{tehtava}
    Ratkaise yhtälöryhmä.
    \begin{align*}
        x+y+z+8 &= 0 \\
        x+y+6 &=0 \\
        x+z-70 &=0
    \end{align*}
    \begin{vastaus}
        $x = 72, \, y = -78, \, z = -2$
    \end{vastaus}
\end{tehtava}

\begin{tehtava}
    Ratkaise yhtälöryhmä.
    \begin{align*}
        x+y+2z+12 &= 0 \\
        2x+2y+3z+1 &=0 \\
        3x-4 &=0
    \end{align*}
    \begin{vastaus}
        $x = \frac{4}{3}, \, y = \frac{98}{3}, \, z = -23$
    \end{vastaus}
\end{tehtava}

\begin{tehtava}
    Ratkaise yhtälöryhmä.
    \begin{align*}
        2x+3y+5z+8 &= 0 \\
        3x+5y+8z &=0 \\
        x+y-1 &=0
    \end{align*}
    \begin{vastaus}
        $x = -\frac{63}{2}, \, y = \frac{65}{2}, \, z = -\frac{17}{2}$
    \end{vastaus}
\end{tehtava}

\begin{tehtava}
	Ratkaise yhtälöryhmä.
	\begin{align*}
		x+w+3 &= 0 \\
		x+y+z &= 0 \\
		y-w-3 &= 0 \\
		w-2z+5 &= 0
	\end{align*}
	\begin{vastaus}
		$x=2, y=-2, z=0, w=-5$
	\end{vastaus}
\end{tehtava}


\begin{tehtava}
	Ratkaise yhtälöryhmä.
	\begin{align*}
		x+2y-z &= 0 \\
		x+3y-w &= 0 \\
		x+y+z+w-4 & = 0 \\
		x+2y+2z+2w-6 &= 0
	\end{align*}
	\begin{vastaus}
		$x=2, y=-\frac13, z=\frac43, w=1$
	\end{vastaus}
\end{tehtava}

\begin{tehtava}
	Ratkaise yhtälöryhmä.
    	\begin{align*}
        	x+y+tz &=1 \\
        	x+ty+z &=1 \\
        	tx+y+z &=1
    	\end{align*}
	\begin{vastaus}
		$x = y = z = \frac{1}{t+2}$, kun $1 \neq t \neq -2$. Kun $t = -2$ yhtälöryhmällä ei ole ratkaisuja.
		Kun $t = 1$ kaikki kolmikot muotoa $x = r$, $y = s$ ja $z = 1-r-s$, jollain reaaliluvuilla $r$ ja $s$ ovat ratkaisuja.
	\end{vastaus}
\end{tehtava}

\begin{tehtava}
	Kolmion sivujen keskipisteet ovat $(1,2)$, $(-1,5)$ ja $(-2,-1)$. Määritä kolmion kärkien koordinaatit.
	\begin{vastaus}
		$(-4,2)$, $(2,8)$ ja $(0,-4)$
	\end{vastaus}
\end{tehtava}


\begin{tehtava}
  Yhden muuttujan lineaarinen yhtälö $ax +b = 0$ määrittää pisteen ($x = \frac{-b}{a}$) lukusuoralla.
  Kahden muuttujan lineaarinen yhtälö puolestaan $ax +by +c = 0$ määrittää kaksiulotteisessa tasossa suoran eli pistejoukon $(x,y) = (x, \frac{-a}{b}x - \frac{c}{b})$.
  Kolmen muuttujan avulla voidaan määrittää kolmiulotteisen \emph{avaruuden} piste $(x, y, z)$.
  Tutki, minkä pistejoukon kolmen muuttujan lineaarinen yhtälö määrittää, toisin sanoen, mitkä kolmiulotteisen avaruuden pisteet $(x, y, z)$ toteuttavat yhtälön $ax +by +cz +d = 0$
  \alakohdat{
    § kun $a = 1, b = 0, c= 0, d=0$?
    § kun $a = 1, b = 1, c= 0, d=1$?
    § kun $a, b, c$ ja $d$ ovat kaikki erisuuria kuin nolla?
  }
  \begin{vastaus}
    \alakohdat{
      § Saadaan yhtälö  $x = 0$ ja siten pistejoukko $(0, y, z)$ ($y$ ja $z$-koordinaattia ei rajoiteta mitenkään, ne voivat olla mitä tahansa reaalilukuja), joka on kolmiulotteisessa $xyz$-avaruudessa $y$- ja $z$-akselien suuntainen taso. 
      § $x + y +1 = 0$ ja $ (x, -x -1, z)$. Yhtälö määrää $xy$-tasossa suoran $y = -x-1$, ja $z$ voi olla mikä tahansa reaaliluku. Saadaan siis $z$-akselin suuntainen taso, joka kulkee $xy$-tason suoran $y = -x-1$ kautta.
      § Kukin muuttuja voidaan kirjoittaa kahden muun avulla, esim. $ z= -\frac{(ax + by + d)}{c}$, jolloin saadaan pistejoukko $(x, y, -\frac{(ax + by + d)}{c})$. Siis kun valitaan mitkä tahansa $x$- ja $y$-koordinaatit, $z$-koordinaatti on määrätty. Yhtälö määrittää edelleen tason.
    }
  \end{vastaus}
\end{tehtava}

\subsubsection*{Sekalaisia tehtäviä}

\begin{tehtava} 
Suorakulmion piiri on $34$ ja lävistäjä $13$. Ratkaise suorakulmion sivut.
    \begin{vastaus}
	Sivut ovat $5$ ja $12$.
    \end{vastaus}
\end{tehtava}

\end{tehtavasivu}
