\begin{tehtavasivu}

\subsubsection*{Opi perusteet}

\begin{tehtava}
Määritä suoran ja ympyrän leikkauspisteet, jos suoran yhtälö, ja ympyrän keskipiste ja säde ovat
\alakohdat{
§ $3x-2y = 1$, $(-1,2)$, $4$
§ $x+1 = 0$, $(-6,-6)$, $10$
§ $x+7y-23 = 0$, $(7,-3)$, $5$
§ $5x+12y-13 = 0$, $(0,0)$, $5$
}
vastaavasti.
\begin{vastaus}
\alakohdat{
§ $(-1,-2)$ ja $(\frac{35}{13},\frac{46}{13})$
§ $(-1,-6+5\sqrt{3})$, $(-1,-6-5\sqrt{3})$
§ Käyrät eivät leikkaa
§ $(\frac{5}{13},\frac{12}{13})$
}
\end{vastaus}
\end{tehtava}

\begin{tehtava}
Määritä kaikki ympyröiden $ (x-3)^2+y^2= 10$, $(x+1)^2+(y-3)^2 = 18$ sekä $(x-6)^2+(y-2)^2 = 8$ leikkauspisteet (pareittain).
\begin{vastaus}
$(\frac{82-9\sqrt{79}}{50},-3\frac{4\sqrt{79}-17}{50})$, $(\frac{82+9\sqrt{79}}{50},3\frac{4\sqrt{79}+17}{50})$,
$(\frac{123-2\sqrt{295}}{26}, 3\frac{10+\sqrt{295}}{26})$
$(\frac{123+2\sqrt{295}}{26}, 3\frac{10-\sqrt{295}}{26})$
$(\frac{16}{5}, \frac{12}{5})$
\end{vastaus}
\end{tehtava}

\begin{tehtava}
Määritä parametrin $k$ arvot siten, että suora $y=x+k$ on ympyrän  $ x^2+y^2= 2$ tangentti.
\begin{vastaus}
$k = \pm sqrt{2} $,
\end{vastaus}
\end{tehtava}


\subsubsection*{Hallitse kokonaisuus}
\begin{tehtava}
	Määritä annetun ympyrän tangentit, jotka kulkevat annetun pisteen kautta.
	\alakohdat{
		§ Piste $(1, -1)$, ympyrä $(x-3)^2 + (y+1)^2 = 2$
		§ Piste $(-4, -2)$, ympyrä $(x+5)^2 + (y+6)^2 = 17$
		§ Piste $(3, 1)$, ympyrä $(x-4)^2 + y^2 = 1$
		§ Piste $(2, 4)$, ympyrä $x^2 + (y-3)^2 = 8$
	}
	\begin{vastaus}
		\alakohdat{
			§ $y=x-2$ ja $y=-x$
			§ $y= -1/4 x -3$
			§ $y=1$ ja $x=3$
			§ Ei ole.
		}
	\end{vastaus}
\end{tehtava}

\begin{tehtava}
Määritä yksikköympyrän $x^2+y^2= 1$ pisteeseen $(x_{0}, y_{0} )$ piirretyn tangentin normaalimuotoinen yhtälö. Entä jos ympyrä on $r$-säteinen?
\begin{vastaus}
$x_0x+y_0y=1 $. $r$-säteisellä ympyrällä $x_0x+y_0y=r^2$
\end{vastaus}
\end{tehtava}

\begin{tehtava}
Yksikköympyrälle $x^2+y^2=1$ piirretään tangentit pisteestä $(0, a)$. Millä $a$:n arvoilla tangentteja on 
\alakohdat{
§ kaksi
§ yksi
§ nolla?
}
Määritä myös tangenttien yhtälöt.
\begin{vastaus}
\alakohdat{
§ $a > 1$ tai $a < -1$
§ $a = \pm1$
§ $ -1 < a < 1 $ 
}
Tangenttien yhtälöt ovat $ y = \pm \sqrt{a^2-1}x+a$
\end{vastaus}
\end{tehtava}

\begin{tehtava}
\alakohdat{
§ Pisteen $P$ etäisyys $O$-keskisestä $r$-säteisestä ympyrästä on $d$ $(d > r) $. Kuinka pitkiä ovat ympyrälle pisteestä $P$ piirretyt tangentit?
§ Yksikköympyrälle ($x^2+y^2 = 1$) ja ympyrälle $(x+3)^2+(y-2)^2 = 2$ piirretään tangentit. Määritä kaikki pisteet $P = (x,y)$, joista piirretyt tangentit ovat yhtä pitkät. (Tätä pistejoukkoa kutsutaan yleensä ympyröiden \emph{radikaaliakseliksi}.)
}

\begin{vastaus}
\alakohdat{
§ $\sqrt{d^2-r^2}$
§ $3x-2y+6$
}
\end{vastaus}
\end{tehtava}

\begin{tehtava}
Määritä kaikkien pisteen $(1,0)$ kautta kulkevien ympyrän $x^2+y^2 = 4$ jänteiden keskipisteiden joukko.
	\begin{vastaus}
		Ympyrä $(x-\frac{1}{2})^2+y^2 = \frac{1}{4}$. Vinkki: Parametrisoi kaikki 			pisteen $(1,0)$ kautta kulkevat suorat, jatutki miten keskipisteiden $x$ ja 		$y$ -koordinaatit riippuvat parametrista.
	\end{vastaus}
\end{tehtava}

\subsubsection*{Sekalaisia tehtäviä}


TÄHÄN TEHTÄVIÄ SIJOITTAMISTA ODOTTAMAAN

\begin{tehtava}
Suoran ulkopuolisesta pisteestä $P$ piirrettyjen tangenttien ja ympyrän sivuamispisteet voidaan määrittää myös monella muulla tavalla:

Pisteen etäisyys suorasta -kaavalla: Jos suora on ympyrän tangentti, sen etäisyyden suorasta on oltava yhtä suuri kuin ympyrän säde.

Määrittämällä tangenttien pituudet: Koska tangentit ja ympyrän säde ovat kohtisuorassa, tangentien pituus voidaan määrittää ympyrän keskipisteen, pisteen $P$ ja sivuamispisteen muodostamasta suorakulmaisesta kolmiosta. Nyt sivuamispisteet voidaan määrittää kahden ympyrän leikkauspisteinä.

Klassiseen tyyliin Thaleen lausetta hyödyntäen: Voidaan osoittaa, että jos $M$ on pisteiden $P$ ja ympyrän keskipisteen $O$ keskipiste, $M$-keskinen ympyrä, joka kulkee pisteiden $P$ ja $O$ kautta leikkaa alkuperäistä ympyrää halutuissa sivuamispisteissä.

Parametrisoimalla kaikki tangentit: Kaikki ympyrän tangentit voidaan myös parametrisoida valitsemalla ympyrältä mielivaltainen piste ja määrittämällä sen kautta kulkeva tangentti (ks. teht ??). Sitten riittää tarkistaa, mitkä tangenteista kulkevat pisteen $P$ kautta.

Malliratkaisun variaatiolla: Parametrisoidaan kaikki $P$:n kautta kulkevat. Nyt saadaan malliratkaisun tavoin yhtälö leikkauspisteille. Sen sijaan, että nyt määritettäisiin, milloin syntyvän toisen asteen yhtälön diskriminantti on nolla, oletetaan, että se on nolla, jolloin leikkauspisteet saadaan määritettyä parametrin funktiona. Nyt riittää enää tarkistaa, milloin nämä pisteet todellakin ovat ympyrällä.

\alakohdat{
§ Jos $P = (0,0)$, $O = (6,4)$ ympyrän säde on $5$, määritä sivuamispisteet näillä viidellä tavalla.
§ Määritä vastaavien tangenttien yhtälöt.
}
\begin{vastaus}
\alakohdat{
§ $(\frac{81-30\sqrt{3}}{26},\frac{54+45\sqrt{3}}{26})$ ja $(\frac{81+30\sqrt{3}}{26},\frac{54-45\sqrt{3}}{26})$
§ $(27-10\sqrt{3})y-(18+15\sqrt{3})x = 0$ ja $(27+10\sqrt{3})y-(18-15\sqrt{3})x = 0$
}
\end{vastaus}
\end{tehtava}

\begin{tehtava}
Kaksi ympyrää leikkaa kohtisuorasti, jos ne leikkaavat, ja niiden leikkauspisteisiin piirretyt tangentit ovat kohtisuorassa. Leikkaavatko ympyrät kohtisuorasti
\alakohdat{
§ $(x+3)^2+(y+3)^2= 13$ ja $(x+2)^2+(y-1)^2 = 5 $
§ $(x-2)^2+y^2 = 16$ ja $(x+3)^2+y^2 = 9$
§ $(x+7)^2+(y-2)^2 = 23$ ja $(x+4)^2+(y+3)^2 = 1$?
}  
\begin{vastaus}
\alakohdat{
§ Eivät
§ Kyllä
§ Eivät
}
\end{vastaus}
\end{tehtava}

\begin{tehtava}
Matti ajaa autolla $(1,0)$-keskistä ympyrärataa, mutta kuin taikaiskusta renkaista lähtee pito ja kauhukseen Matti alkaa luisua irtoamispisteestä ympyräradan tangentin suuntaan. Matti saa pysäytettyä autonsa pisteeseen $(5,5)$. Kuinka pitkä ympyrärata on, jos se on yhtä pitkä kuin Matin luisuma matka?
\begin{vastaus}
$\frac{2\sqrt{41}\pi}{\sqrt{4\pi^2+1}}$
\end{vastaus}
\end{tehtava}

\begin{tehtava}
Todista analyyttisen geometrian keinoin tangenttien tärkeät perusominaisuudet
\alakohdat{
§ Ympyrälle piirretty tangentti ja tangentin sivuamispisteeseen piirretty säde ovat kohtisuorassa.
§ Tangentit ovat säteen etäisyydellä ympyrän keskipisteestä.
§ Suoran ulkopuolisesta pisteestä piirretyt kaksi tangenttia ovat yhtä pitkiä.
}
Vinkki: Yleisyyttä menettämättä voidaan ympyrä asettaa origokeskiseksi ja yksisäteiseksi, ja tangentti valita kulkemaan sopivan pisteen kautta.
\begin{vastaus}
Vinkki on hyvä
\end{vastaus}
\end{tehtava}

\begin{tehtava}
\alakohdat{
§ Määritä ympyrän $(y-r)^2+x^2 = r^2$ pisteen $(1,0)$ kautta kulkevat tangentit.
§ Minkä käyrän muodostavat vastaavat sivuamispisteet, kun $r$ saa kaikki positiiviset kokonaislukuarvot.
}
	\begin{vastaus}
		\alakohdat{
			§ Suorat $y = 0$ ja $2rx-(r^2-1)y-2r = 0$.
			§
		}
	\end{vastaus}
\end{tehtava}


%%%Sarjassamme liian helppoja todistuksia:
\begin{tehtava}
Edellä esimerkissä todettiin, että suoralla ja ympyrällä voi olla nolla, yksi tai kaksi leikkauspistettä riippuen siitä, montako ratkaisua on toisen asteen yhtälöllä joka syntyy kun kirjoitetaan ympyrän ja suoran yhtälöt yhtälöpariksi. Totea, että \emph{minkä tahansa} ympyrän ja suoran yhtälöistä todellakin aina saadaan toisen asteen yhtälö.
\begin{vastaus}
Ympyrän ja suoran yleiset yhtälöt.
\end{vastaus}
\end{tehtava}

\begin{tehtava}
Määritä ympyröiden $x^2+y^2 = 4$ ja $x^2+(y-4)^2 = 1$ yhteiset tangentit.

\begin{vastaus}
$\pm\sqrt{15}x+y-8= 0$ ja $\pm\sqrt{7}x+3y-8$. Vinkki: Hyödynnä pisteen etäisyys suorasta -kaavaa.
\end{vastaus}
\end{tehtava}

\begin{tehtava}
Määritä kaikkien niiden janojen keskipisteiden joukko, joiden toinen päätepiste on suoralla $x = 2$, toinen ympyrällä $x^2+y^2 = 2$, ja jotka kulkevat (tai niiden jatke kulkee) origon kautta. (Käyrä tunnetaan nimellä \textit{Conchoid of Nicomedes})
	\begin{vastaus}
		$4(x-1)^2(x^2+y^2) = x^2$
	\end{vastaus}
	
\end{tehtava}

\end{tehtavasivu}
