\laatikko{
KIRJOITA TÄHÄN LUKUUN

\luettelo{
§ TEHTY - ympyrän ja suoran leikkauspisteiden ratkaiseminen 
§ ympyrän tangetin määrittäminen sekä kehällä olevan (kohtisuorassa sädettä vastaan TEHTY) että
sen ulkopuolisen pisteen kautta (kaksi tapaa: pisteen etäisyys suorasta -kaava (TEHTÄVÄKSI?) tai diskriminantti = 0 TEHTY)
§ TEHTY kahden ympyrän leikkauspisteiden ratkaiseminen yhtälöparilla
}

KIITOS!}

\begin{esimerkki}
Määritä suorien $x+2y-3=0$ ja ympyrän $(x-1)^2+(y+1)^2=4 $ leikkauspisteet.

\begin{esimratk}

Ympyrän ja suoran leikkauspisteet ovat ne pisteet $(x, y)$, jotka ovat sekä suoralla että ympyrällä, eli toteuttavat molempien yhtälöt, eli yhtälöryhmän
$$\left\{    
    \begin{array}{rcl}
        x+2y-3 &=&0 \\
        (x-1)^2+(y+1)^2 &=&4 \\
    \end{array}
    \right.$$
Vaikka yhtälö ei olekaan tutun lineaarinen, myös sitä voi lähestyä sijoitusmenetelmällä. Ratkaistaan ensimmäisestä yhtälöstä $x$ ja saadaan
\[
x = -2y+3
\]
Kun tämä sijoitetaan toiseen yhtälöön ja kerrotaan auki syntyy toisen asteen yhtälö $y$:n suhteen:
\begin{align*}
(-2y+3-1)^2+(y+1)^2=4 \\
(-2y+2)^2+(y+1)^2=4 \\
(-2y)^2-2\cdot 2y\cdot 2 +2^2+y^2+2\cdot y+1^2=4 \\
4y^2-8y+4+y^2+2y+1-4 = 0 \\
5y^2-6y+1 = 0
\end{align*}
Ratkaisukaavalla
\[
y = \frac{-(-6)\pm\sqrt{(-6)^2-4\cdot 5\cdot 1}}{2\cdot5}
\]
eli
\begin{align*}
y = \frac{6\pm\sqrt{16}}{10} \\
y = 1 \vee y = \frac{1}{5}
\end{align*}
Kun nämä $y$:n arvot sijoitetaan suoran yhtälöön saadaan vastaavast $x$:n arvot:
\begin{align*}
x = -2\cdot 1+3 \vee x = -2\cdot\frac{1}{5}+3 \\
x = 1 \vee x = \frac{13}{5}
\end{align*}
eli saatiin 2 ratkaisua: $(x, y) = (1, 1)$ ja $(x, y) = (\frac{13}{5}, \frac{1}{5})$. Tarkistamalla on hyvä vielä todeta, että pisteet todella ovat leikkauspisteitä.

\begin{kuva}
    kuvaaja.pohja(-3, 5, -4, 4, korkeus = 4, nimiX = "$x$", nimiY = "$y$", ruudukko = True)
    kuvaaja.piirraParametri("2*cos(t)+1","2*sin(t)-1", a = 0, b = 2*pi)
    kuvaaja.piirra("(-x+3)/2")
	
\end{kuva}
\begin{esimvast}
Leikkauspisteet ovat $(1, 1)$ ja $(\frac{13}{5}, \frac{1}{5})$.
\end{esimvast}

\end{esimratk}
\end{esimerkki}

Esimerkin avulla huomattiin, että suoralla ja ympyrällä voi olla kaksi leikkauspistettä. Suorasta ja ympyrästä riippuen päädytään toisen asteen yhtälöön, jolla on joko $0$, $1$ tai $2$ ratkaisua. Jos leikkauspisteitä on tasan yksi, sanotaan, että suora sivuaa ympyrää tai suora on ympyrän \termi{tangentti}{tangentti}.

Kuva ympyrästä ja kolmesta suorasta; yksi tangentti, yksi leikkaa kahdessa pisteessä ja yksi ei leikkaa ympyrää.

\laatikko{Suoralla ja ympyrällä on nolla, yksi tai kaksi leikkauspisteitä.

Suora on ympyrän \emph{tangentti}, jos suoralla ja ympyrällä on tasan yksi yhteinen piste.
}

\begin{esimerkki}

Määritä $(2, 2)$-keskisen 5-säteisen ympyrän pisteen $(-5, 3)$ kautta kulkevat tangentit.

\begin{esimratk}
Suora on ympyrän tangentti, jos sillä ja ympyrällä on tasan yksi yhteinen piste. Jos pisteen $(-5, 3)$ suora ei ole $y$-akselin suuntainen, se voidaan esittää muodossa
\[
y-3 = k(x-(-5))
\]
eli
\[    
y = kx+5k+3.
\]
(Oletus voidaan tehdä, sillä selvästi nähdään ettei pisteen $(-5, 3)$ kautta kulkeva pystysuora suora ole kyseisen ympyrän tangentti.)

Kun lisäksi muistamme ympyrän yhtälön $(x-x_0)^2+(y-y_0)^2 = r^2$, suoran ja ympyrän leikkauspisteille saadaan yhtälöpari

\[
\left\{    
    \begin{array}{rcl}
        y &=& kx+5k+3 \\
        (x-2)^2+(y-2)^2 &=& 25 
\end{array}
    \right.
\]
    
Sijoitetaan ensimmäisen yhtälön lauseke $y$:lle toiseen yhtälöön ja saadaan toisen asteen yhtälö $x$:n suhteen
\begin{align*}
(x-2)^2+(kx+5k+3-2)^2&=25 \\
(x-2)^2+(kx+5k+1)^2&=25 \\
x^2-4x + 4+(kx)^2+kx\cdot 5k+kx \quad &\\
+5k\cdot kx+(5k)^2+5k+kx+5k+1& = 25\\
(k^2 +1)x^2+(10k^2+2k-4)x+25k^2+10k-20& = 0. \\
\end{align*}
Tällä yhtälöllä on $x$:n suhteen tasan yksi ratkaisu kun polynomin diskriminantti on nolla, eli
\begin{align*}
D &= b^2 -4ab \\
&= (10k^2+2k-4)^2-4\cdot(k^2 +1)\cdot(25k^2+10k-20) \\ 
&= 100k^4+20k^3-40k^2 + 20k^3 + 4k^2 -8k -40k^2 -8k +16 - 4(k^2 +1)(25k^2+10k-20)\\
&= 100k^4 + 40k^3 - 76k^2 - 16k + 16 - 4(25k^4+10k^3-20k^2 + 25k^2+10k-20) \\
&= 100k^4 + 40k^3 - 76k^2 - 16k + 16  - 100k^4 - 40k^3 - 20k^2 - 40k + 80 \\
& = -96k^2-56k+96 \\
&=-8(12k^2+7k-12) = 0.
\end{align*}
Toisen asteen yhtälön ratkaisukaavalla saadaan
\begin{align*}
k &= \frac{-7\pm \sqrt{7^2-4\cdot 12\cdot (-12)}}{2\cdot 12} \\
k &= \frac{-7\pm \sqrt{625}}{24} \\
k &= \frac{-7\pm 25 }{24}, \\
\end{align*}
joten
\begin{align*}
k = \frac{18}{24} =  \frac{3}{4} \; &\textrm{tai} \; k = -\frac{32}{24} = -\frac{4}{3}
\intertext{Kulmakertoimia vastaavat siis tangenttisuorat}
y = \frac{3}{4}x+5\cdot \frac{3}{4}+3 \; &\textrm{ja} \; y = -\frac{4}{3}x+5\cdot \Big(-\frac{4}{3}\Big)+3 \\
y = \frac{3}{4}x+\frac{27}{4} \; &\textrm{ja} \; y = -\frac{4}{3}x-\frac{11}{3}
\end{align*}
\end{esimratk}
\begin{esimvast}
Suorat $y = \frac{3}{4}x+\frac{27}{4}$ ja $y = -\frac{4}{3}x-\frac{11}{3}$
\end{esimvast}
\end{esimerkki}

Vastaavalla tavalla voidaan määrittää myös kahden ympyrän leikkauspisteet

\begin{esimerkki}
Määritä ympyröiden $(x-1)^2+(y+5)^2 = 13$ ja $(x+2)^2+(y+4)^2 = 17$ leikkauspisteet.

\begin{esimratk}
Ympyröiden leikkauspisteet toteuttavat yhtälöparin
\[
\left\{    
    \begin{array}{rcl}
        (x-1)^2+(y+5)^2 = 13 \\
        (x+2)^2+(y+4)^2 = 17 \\
    \end{array}
    \right.
\]
Kun binomien neliöt kerrotaan auki, sievennetään ja yhtälöt vähennetään toisistaan saadaan $x$:n ja $y$:n välille ensimmäisen asteen yhtälö
\[
\left\{    
    \begin{array}{rcl}
        x^2-2x+1+y^2+10y+25 = 13 \\
        x^2+4x+4+y^2+8y+16 = 17 \\
    \end{array}
    \right.
\]
\[
\left\{    
    \begin{array}{rcl}
        x^2-2x+y^2+10y= -13 \\
        x^2+4x+y^2+8y= -3 \\
    \end{array}
    \right.
\]
joten
\[
-6x+2y=-10
\]
Tämä suoran yhtälö vastaa oikeastaan ympyröiden leikkauspisteiden kautta kulkevaa suoraa. Tästä voidaan ratkaista $y$ $x$:n suhteen ja sijoittaa jompaan kumpaan alkuperäisistä yhtälöistä.
\begin{align*}
y = 3x-5 \\
(x-1)^2+((3x-5)+5)^2 = 13 \\
x^2-2x+1+(3x)^2 = 13 \\
10x^2-2x-12 = 0 \\
5x^2-x-6 = 0 \\
\end{align*}
Nyt leikkauspisteiden $x$-koordinaatit voidaan ratkaista toisen asteen yhtälön ratkaisukaavalla
\begin{align*}
x &= \frac{-(-1)\pm\sqrt{(-1)^2-4\cdot 5\cdot (-6)}}{2\cdot 5} \\
&= \frac{1\pm\sqrt{121}}{10} \\
&= \frac{1\pm 11}{10} \\
\end{align*}
eli
\[
x =  \frac{6}{5} \textrm{  tai  } x = -1
\]
Nyt suoran yhtälöstä voidaan ratkaista vastaavat $y$:n arvot.
\begin{align*}
y = 3\cdot \frac{6}{5}-5 &\textrm{  tai  } y = 3\cdot (-1)-5 \\
y = -\frac{7}{5}x &\textrm{  tai  } y = -8
\end{align*}
\end{esimratk}
\begin{esimvast}
Ympyröiden leikkauspisteet ovat $(\frac{6}{5}, -\frac{7}{5})$ ja $(-1,-8)$.
\end{esimvast}
\end{esimerkki}

Jälleen esimerkistä nähdään, että riippuen syntyvän toisen asteen yhtälön diskriminantista kahdella ympyrällä voi olla $0$, $1$ tai $2$ leikkauspistettä.

%%FIXME: kuvan voisi päivittää liittymään esimerkkiin?
\begin{kuva}
    kuvaaja.pohja(-7, 5, -11, 3, korkeus = 5, nimiX = "$x$", nimiY = "$y$", ruudukko = True)
    kuvaaja.piirraParametri("3.606*cos(t)+1","3.606*sin(t)-5", a = 0, b = 2*pi)
    kuvaaja.piirraParametri("4.123*cos(t)-2","4.123*sin(t)-4", a = 0, b = 2*pi)
	
\end{kuva}

Ympyrän tangentti saadaan yksikäsitteisesti myös kun tiedetään ympyrän kehän piste jossa tangentti sivuaa ympyrää:

\begin{esimerkki}
Mikä on ympyrää $(x-3)^2 + (y -1)^2 = 18$ pisteessä $(6,4)$ sivuavan tangentin yhtälö?
\begin{esimratk}
Koska tangentti kulkee pisteen $(6,4)$ kautta, jolloin sen yhtälö on 
\begin{align*}
y - 4 &= k_1(x-6) \\
y &= k_1x -6k_1 +4.
\end{align*}
Voisimme sijoittaa tämän edellisten esimerkkien tapaan ympyrän yhtälöön, mutta toinen (tässä tapauksessa helpompi) tapa on huomata tangentin olevan kohtisuorassa ympyrän sädettä eli keskipisteen $(3, 1)$ (jonka voimme lukea suoraan ympyrän yhtälöstä) ja pisteen $(6,4)$ välistä janaa vasten. Nämä kaksi pistettä määrittävät suoran, jonka kulmakerroin $k_2$ on
\begin{align*}
k_2 = \frac{y_2 - y_1}{x_2 - x_1} = \frac{4-1}{6-3} = \frac{3}{3} = 1.
\end{align*}
Koska suoran ja sen normaalin kulmakertoimien tulo on $k_1 k_2 = -1$, saadaan tangentin kulmakertoimeksi $k_1 = k_1 \cdot 1 = -1$, jonka voimme sijoittaa tangentin yhtälöön.
\end{esimratk}
\begin{esimvast}
Tangentin yhtälö on $y = -x +10$.
\end{esimvast}
\end{esimerkki}

%%%KUVA? %%%https://www.wolframalpha.com/input/?i=y%20%3D%20%2Dx%20%2B10%2C%20(x%2D3)%5E2%20%2B%20(y%20%2D1)%5E2%20%3D%2018
