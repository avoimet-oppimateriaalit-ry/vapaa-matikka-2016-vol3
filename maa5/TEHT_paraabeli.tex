\begin{tehtavasivu}

\subsubsection*{Opi perusteet}

\begin{tehtava}
    Kuinka monta leikkauspistettä voi olla paraabelilla ja
    \alakohdat{
        § suoralla,
        § ympyrällä,
        § toisella paraabelilla?
    }
    \begin{vastaus}
        \alakohdat{
            § 0--2
            § 0--4
            § 0--4
        }
    \end{vastaus}
\end{tehtava}

\begin{tehtava}
Ratkaise paraabelien $y=5(x-2)^2$  ja $y=-x^2+2x+4$ leikkauspisteet?
\begin{vastaus}
%tulee toisen asteen yhtälö ratkaistavaksi
% http://www.wolframalpha.com/input/?i=y%3D5%28x-2%29%5E2%2C+y%3D-x%5E2%2B2x%2B4
$(1, 5)$ ja $(\frac{8}{3}, \frac{20}{9})$
\end{vastaus}
\end{tehtava}

\begin{tehtava}
%tehtävä helpottuu, koska vakiotermin saa suoraan
Määritä sen ylöspäin aukeavan paraabelin yhtälö, joka kulkee pisteiden $(-5, 6)$, $(0, -4)$ ja $(1, 0)$ kautta.
\begin{vastaus}
% http://www.wolframalpha.com/input/?i=y%3D+x%5E2%2B3x-4+at+x%3D%7B-5%2C+0%2C+1%7D
$y= x^2+3x-4$
\end{vastaus}
\end{tehtava}

\begin{tehtava}
Määritä sen alaspäin aukeavan paraabelin yhtälö, joka kulkee pisteiden $(-5, 6)$, $(0, -4)$ ja $(1, 0)$ kautta.
\begin{vastaus}
% http://www.wolframalpha.com/input/?i=y%3D+x%5E2%2B3x-4+at+x%3D%7B-5%2C+0%2C+1%7D
$y= x^2+3x-4$
\end{vastaus}
\end{tehtava}

\begin{tehtava}
Millaisella käyrällä ovat ympyröiden keskipisteet, kun ympyrät kulkevat pisteen $(0, 0)$ kautta ja sivuavat suoraa $x=4$?
\begin{vastaus}
%keskipiste (x, y)
% x< 4 
% kulkee origon kautta, joten (0-x)^2+(0-y)^2=r^2
% etäisyys suorasta |x-4| = r
$x=-\frac{1}{8}y^2+2$
\end{vastaus}
\end{tehtava}

\begin{tehtava}
Millä vakion $a$ arvolla suora $y=x$ on paraabelin $y=x^2-3x+a$ tangentti?
\begin{vastaus}
% yhtälöllä x = x^2-3x+a pitäisi olla tasan yksi ratkaisu 0 = (x-2)^2-4+a
% http://www.wolframalpha.com/input/?i=y%3D+x%5E2-3x%2B4%2C+y%3Dx
$a=4$
\end{vastaus}
\end{tehtava}

\subsubsection*{Hallitse kokonaisuus}


\begin{tehtava}
Määritä kaikki ne pisteet, jotka ovat yhtä etäällä ympyrästä $x^2+y^2 = 4$ ja suorasta
	\alakohdat{
		§ $y = -4$
		§ $y = -2$
		§ $y = 0$
		§ $y = -1$
	}
	\begin{vastaus}
		\alakohdat{
		§ $y = \frac{x^2}{12}-3$
		§ $y = \frac{x^2}{8}-2$ tai $x = 0$ ja $y \leq 0$
		§ $y = \pm \frac{1}{4}(x^2-4$
		§ $y = \frac{1}{2}(1-x^2)$ tai $y = \frac{1}{6}(x^2-9)$
		}
	\end{vastaus}
\end{tehtava}

\begin{tehtava}
Yhdensuuntaiset suorat $l_1$ ja $l_2$ leikkaavat paraabelia $y = ax^2+bx+c$ pisteissä $P_1$ ja $Q_1$, sekä $P_2$ ja $Q_2$, vastaavasti. Todista, että janojen $P_1 Q_1$ ja $P_2 Q_2$ keskipisteiden kautta kulkeva suora on $y$-akselin suuntainen.
	\begin{vastaus}
		Vinkki: Riittää todeta, että leikkauspisteiden $x$-koordinaattien keskiarvo ei riipu 		leikkaavan suoran vakiotermistä.
	\end{vastaus}
\end{tehtava}

\subsubsection*{Sekalaisia tehtäviä}

TÄHÄN TEHTÄVIÄ SIJOITTAMISTA ODOTTAMAAN

\begin{tehtava}
Mikä on sen käyrän yhtälö, jonka kukin piste on yhtä etäällä suorasta $y=0$ ja pisteestä $(-3, 1)$.
\begin{vastaus}
% http://www.wolframalpha.com/input/?i=+sqrt%28%28-3-x%29%5E2+%2B+%281-y%29%5E2%29%3D+abs%280-y%29
$y = \frac{1}{2}x^2+3 x+5$
\end{vastaus}
\end{tehtava}

\begin{tehtava}
%%% ONKO JOHTOSUORA NIIN OLEELLINEN KÄSITE, ETTÄ KÄYTETÄÄN TEHTÄVISSÄ?
%% tehtävänhän voi kirjoittaa ilman tuota termiä
Mikä on sen paraabelin yhtälö, jonka polttopiste on $(2, 3)$ ja johtosuora $y=1$?
\begin{vastaus}
% http://www.wolframalpha.com/input/?i=+sqrt%28%282-x%29%5E2+%2B+%283-y%29%5E2%29%3D+abs%281-y%29
$y = \frac{1}{4}x^2-x+3$
\end{vastaus}
\end{tehtava}

\begin{tehtava}
Mitkä ovat suoran $x+y=6$ ja paraabelin $y=4x^2-3x$ leikkauspisteet?
\begin{vastaus}
% http://www.wolframalpha.com/input/?i=y%3D4x%5E2-3x%2C+x%2By%3D6
$x = -1$, $ y = 7$ tai $x = \frac{3}{2}$, $y = \frac{9}{2}$
\end{vastaus}
\end{tehtava}

%%%%%TÄMÄ JA SEURAAVA PITÄISI SIIRTÄÄ VASEMMALLE/OIKEALLE AUKEAVIEN PARAABELIEN JÄLKEEN?
\begin{tehtava}
Millaisella käyrällä ovat ympyröiden keskipisteet, kun ympyrät kulkevat pisteen $(0, 0)$ kautta ja sivuavat suoraa $x=4$?
\begin{vastaus}
%keskipiste (x, y)
% x< 4 
% kulkee origon kautta, joten (0-x)^2+(0-y)^2=r^2
% etäisyys suorasta |x-4| = r
$x=-\frac{1}{8}y^2+2$
\end{vastaus}
\end{tehtava}



\begin{tehtava}
Määritä ne paraabelin $y=x^2-1$ pisteet, jotka ovat yhtä kaukana pisteistä $(4, 4)$ ja $(4, 2)$?
\begin{vastaus}
%suoran y=3 ja paraabelin leikkauspisteet
$x=-2$, $y=3$ ja $x=2$, $y=3$
\end{vastaus}
\end{tehtava}

\begin{tehtava}
Määritä ne paraabelin $y=x^2-1$ pisteet, jotka ovat yhtä kaukana pisteistä $(4, 4)$ ja $(3, 3)$?
\begin{vastaus}
%pisteiden keskipisteen (3,5; 3,5) kautta kulkeva suora y=-x+7
%suoran  ja paraabelin leikkauspistee
% http://www.wolframalpha.com/input/?i=y%3Dx%5E2-1%2C+y%3D-x%2B7
$x = -\frac{1+\sqrt{33}}{2}$,   $y = \frac{15+\sqrt{33}}{2}$ tai $x = -\frac{\sqrt{33}-1}{2}$,   $y = \frac{15-\sqrt{33}}{2}$
\end{vastaus}
\end{tehtava}



\begin{tehtava}
% VAIKEA
Määritä kaksi sellaista paraabelia, että niillä on täsmälleen kolme yhteistä pistettä
\begin{vastaus}
%idea valitaan suora, joka on tangetti kummallekin paraabelille esim. y=x ja kumpikin paraabeli sivuaa suoraa samassa kohtaa
% http://www.wolframalpha.com/input/?i=y%3Dx%5E2-x%2C+x%3D2y%5E2-y
Esimerkiksi $y=x^2-x$ ja  $x=2y^2-y$
\end{vastaus}
\end{tehtava}

\begin{tehtava}
Millä parametrin $a$ arvoilla paraabelin $y=x^2-ax+a$ huippu on $x$-akselilla?
\begin{vastaus}
% esim. neliöksi täydentäminen y=(x-a/2)^2 -a^2/4 +a, josta -a^2/4 +a =0
$a=0$ ja $a=4$
\end{vastaus}
\end{tehtava}

\begin{tehtava}
Määritä vakio $a$ siten, että lausekkeen $2x^2+12x+a$ pienin arvo on 10.
\begin{vastaus}
% esim. neliöksi täydentäminen 2x^2+12x+a= 2((x+3)^2-9+a/2), josta 2(-9+a/2)=10
$a=28$
\end{vastaus}
\end{tehtava}

\begin{tehtava}
Ympyrän lisäksi myös muille tasokäyrille voi määritellä tangentin. Paraabelin tangentti on suora, joka leikkaa paraabelia tasan yhdessä pisteessä. Määritä paraabelin $y = x^2$ pisteeseen $(x_0, x_0^2)$ piirretyn tangentin yhtälö.
	\begin{vastaus}
		$y = 2x_0x-x_0^2$
	\end{vastaus}
\end{tehtava}

\begin{tehtava}
Paraabelin  $y = x^2$ kahteen pisteeseen piirretyt tangentit ovat kohtisuorassa. Todista, että tangenttien leikkauspiste on paraabelin johtosuoralla.
	\begin{vastaus}
		Vinkki: Voit hyödyntää edellisen tehtävän tulosta, ja suorien kohtisuoruusehtoa.
	\end{vastaus}
\end{tehtava}

\begin{tehtava}
Kaksi yhdensuuntaista suoraa $l_1$ ja $l_2$ leikkaavat paraabelin $y = x^2$ pisteissä $P_1$ ja $Q_1$, sekä $P_2$ ja $Q_2$, vastaavasti. Pisteistä $P_1$ ja $Q_1$ piirretyt tangentit leikkaavat toisensa pisteessä $R_1$, ja $P_2$ ja $Q_2$ piirretyt tangentit toisensa pisteessä $R_2$. Todista, että suora $R_1R_2$ on $y$-akselin suuntainen.
	\begin{vastaus}
		Vinkki: Miten kahdesta eri pisteestä piirrettyjen tangenttien leikkauspisteiden $x$-koordinaatti riippuu sivuamispisteiden $x$-koordinaateista?
	\end{vastaus}
\end{tehtava}

\begin{tehtava}
Määritä paraabelin $y = 2x^2+x-3$ polttopiste ja johtosuora.
	\begin{vastaus}
		$(-\frac{1}{4},-3)$ ja $y = \frac{13}{4}$
	\end{vastaus}
\end{tehtava}

\begin{tehtava}
Paraabelin $y = x^2$ kaarevuutta origossa voidaan tutkia ympyröiden avulla. Ympyrän $x^2+(y-r)^2=r^2$ keskipiste on $y$-akselilla ja se kulkee origon kautta. Millä $r$:n arvoilla ympyrällä ja paraabelilla on
\alakohdat{
§ yksi
§ kolme leikkauspistettä?
§ Mikä on suurin $r$, jolla leikkauspisteitä on tasan yksi? Tämän voidaan ajatella olevan paraabelin kaarevuussäde origossa.
}
\begin{vastaus}
\alakohdat{
§ $0 < r \geq \frac{1}{2}$
§ $\frac{1}{2} < r $
§ $r = \frac{1}{2}$
}
\end{vastaus}
\end{tehtava}

\end{tehtavasivu}
