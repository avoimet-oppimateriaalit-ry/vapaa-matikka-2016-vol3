\begin{tehtavasivu}

\subsubsection*{Opi perusteet}

\begin{tehtava}
Arkkitehti Guggenheim suunnittelee uuteen taidemuseoon kahta paraabelin muotoista holvikaarta. Holvikaarten leveys on 5 metriä ja korkeus 7 metriä, ne ovat puolen metrin päässä toisistaan ja ne ovat sijoitettu symmetrisesti (y-akseliin nähden) julkisivulle. Määritä holvikaarten yhtälöt.
\begin{vastaus}
%holvikaaret puoli metriä toisistaan, siis etäisyys y-akselista 0,25m.
%Tässä ensimmäinen piste, toinen piste on 5,25m päässä, ja kolmas on (2,75; 7). %Riittää määrittää vain yksi paraabeli, ja toisen saa x_0:n vastaluvusta.
$y-7 = -\frac{6,25}{7}(x - 2,75)^2$ ja $y-7 = -\frac{6,25}{7}(x + 2,75)^2$
\end{vastaus}
\end{tehtava}



\subsubsection*{Hallitse kokonaisuus}

\subsubsection*{Sekalaisia tehtäviä}

TÄHÄN TEHTÄVIÄ SIJOITTAMISTA ODOTTAMAAN

%%% Hei, ei näiden todistustehtävien tarvitse olla vaikeita! Testataan onko luettu myös ymmärretty. :D
\begin{tehtava}
Osoita, että jos toisen asteen yhtälöllä on kaksi nollakohtaa, sen kuvaajan paraabelin huipun $x$-koordinaatti on nollakohtien puolivälissä.
\begin{vastaus}
    \emph{Vihje:} Toisen asteen yhtälön ratkaisukaava.
\end{vastaus}
\end{tehtava}

%%% Jatkoa edellisen teemaan. Tämäkin onnistuu varmaan kuinka monella tavalla.
\begin{tehtava}
Osoita miksi paraabeli $y = ax^2 +bx +c$ on symmetrinen huippunsa kautta kulkevan $y$-akselin suuntaisen suoran suhteen.
\begin{vastaus}
    \emph{Vihje:} Edelleen, toisen asteen yhtälön ratkaisukaava.
\end{vastaus}
\end{tehtava}

%%% Helppoa intuitiota funktion ääriarvopisteiden etsimiseen
\begin{tehtava}
Usein huomataan, että halutaan tietää missä pisteessä jokin funktio saa pienimmän tai suurimman arvonsa.

Onko seuraavilla funktioilla suurinta tai pienintä arvoa? Jos on, mikä se on, ja millä $x$:n arvolla se saavutetaan?
\alakohdat{
    § $f(x) = 4x^2 - 8x + 8$
    § $f(x) = -x^2 - 3x - 1$
    § $f(x) = (x-2)(x-3)$
}

Kurssilla MAA??? (derivaattakurssi?) tutustutaan tarkemmin funktion ääriarvojen (pieninten ja suurinten arvojen) etsimiseen, erityisesti myös silloin kun funktio ei ole toisen asteen polynomi.
	\begin{vastaus}
	\alakohdat{
	    § Pienin arvo $4$ pisteessä $x = 1$ (funktion kuvaaja on ylöspäin aukeava paraabeli).
	    § Suurin arvo $\frac{5}{4}$ pisteessä $\frac{-3}{2}$.
	    § Pienin arvo $\frac{-1}{4}$ pisteessä $\frac{5}{2}$.
	}
	\end{vastaus}
\end{tehtava}

%%%Jatkona edelliseen, tähän pari soveltavampaa 'geometrista' ääriarvotehtävää jossa funktio toista astetta

%%%Ei tämäkään välttämättä vaikea:
\begin{tehtava}
Käytettävissä on 100 tulitikkua, joista jokainen on 4~cm pitkä, ja niistä halutaan muodostaa pinta-alaltaan mahdollisimman suuria kuvioita. Tikkuja saa katkaista mistä kohtaa tahansa.
\alakohdat{
    § Kuinka suuren alan peittää tikuista rajattu mahdollisimman suuri suorakulmio, ja mitkä ovat sen sivujen pituudet?
    § Muuten sama kuin edellä, paitsi että tikuista rajataan kaksi samankokoista suorakulmioita joilla on yksi yhteinen tikuista tehty sivu, ja halutaan suorakulmioiden yhteispinta-ala mahdollisimman suureksi?
    § Sama kuin (b), paitsi että suorakulmioiden ei tarvitse olla samankokoisia.
}
	\begin{vastaus}
	\alakohdat{
	    %%%Saattaa huomata ilman laskemistakin että suurin ala on neliöllä. 
	    %%%Lisäksi huomattava ettei kysytty alaa tikkuina!
        § Neliö, jonka sivun pituus 100~cm ja pinta-ala 1~m$^2$
        %%%Ei kovin vaikea edellisen sovellus, kunhan huomaa millaista kuviota tarkoitetaan.
        § Kummankin lyhyempi sivu 50~cm ja pitempi sivu $66\frac{2}{3}$~cm$^2$. Yhteispinta-ala $\frac{2}{3}$~m$^2$.
        %%%Kompa. Sillä missä kohtaa sijaitsee suuren suorakulmion kahdeksi jakava sivu ei ole merkitystä pinta-alan tai sivujen yhteispituuksien kannalta!
        § Sama kuin edellisessä.
    }
	\end{vastaus}
\end{tehtava}

\begin{tehtava}
Paraabelin ja suoran välinen etäisyys lyhyimmän mahdollisen paraabelia ja suoraa yhdistävä janan pituus. Määritä paraabelin $y = x^2$ etäisyys suorasta $y-x+1 = 0$.
\begin{vastaus}
	$\frac{3}{4\sqrt{2}}$. Vinkki: Mikä on pisteen $(x_0, x_0^2)$ etäisyys suorasta? Miten lauseke 
minimoidaan (kts. edelliset tehtävät)?
\end{vastaus}
\end{tehtava}

%%%Hieman vaikeampi muunnos edellisestä olisi jos tikkujen katkaisemista rajoitetaan...?

%%%Sitten lukion fysiikkaan liittyviä tehtäviä:
%%%*Putoavan kappaleen sijainti
%%%*Tykinkuulan/pallon/tjsp heitto/lento(/...)rata

\end{tehtavasivu}
