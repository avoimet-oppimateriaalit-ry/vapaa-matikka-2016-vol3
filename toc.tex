\renewcommand{\luku}[2]{\section{#2} \lukufilter{#1}{\input{maa5/TEORIA_#1} \input{maa5/TEHT_#1}}} % luku
\renewcommand{\nluku}[2]{\section*{#2} \addcontentsline{toc}{section}{#2} \lukufilter{#1}{\input{maa5/#1}}} % numeroimaton luku

\newpage
\nosa{MAA5 -- Analyyttinen geometria}
\lukufilter{#1}{\subsection*{Lisälukemistoa}

Helsingin yliopiston perinteinen oppimateriaali:

\qrlinkki{http://www.math.helsinki.fi/kurssit/difint11/1999/all.pdf}{Hurri-Syrjänen, Ritva: Differentiaali- ja integraalilaskenta I.1}
}
% Taustaa-osion jälkeen pari outoa riviä

\osa{Esitietoja}
    \luku{itseisarvo}{Itseisarvo ja itseisarvoyhtälöt}
    \luku{epayhtalo}{Itseisarvoepäyhtälöt}
    % itseisarvoepäyhtälöt
    \luku{koordinaatisto}{Koordinaatisto ja yhtälön kuvaaja}
    % yleistä käyristä, esim. Kartesiuksen lehdestä jotain
    % kahden pisteen välinen etäisyys (Pythagoraan lauseella)

\osa{Suorat ja lineaariset yhtälöryhmät}
    \luku{suora}{Suoran yhtälö}
    % ratkaistu muoto y = kx + b, kulmakerroin ja vakiotermi
	% nollakohdat ja leikkauspisteet
	% vaaka- ja pystysuorat
    \luku{suora-esitykset}{Suoran yhtälön muut muodot}
    % esitys y-y_0=k(x-x_0)
	% esitys ax + by + c = 0 (normaalimuoto)
    \luku{pisteen-etaisyys}{Pisteen etäisyys suorasta}
	% kaava pisteen etäisyydelle suorasta
    \luku{suora-asema}{Suorien keskinäinen asema}
	% suorien keskinäinen asema, yhdensuuntaiset suorat
	% suoralle ja sen normaalille k_1 * k_2 = -1
    \luku{yhtaloryhma}{Lineaariset yhtälöryhmät}
	% sijoitusmenetelmä
	% yhtälöiden laskeminen yhteen
	% ratkaisujen määrä

\osa{Toisen asteen käyrät}
	\luku{ympyra}{Ympyrä}
	% ympyrän yhtälö määritelmästä
	% ensin origokeskeinen
	% keskipiste ja säde muissa tapauksissa neliöksi täydentämällä
	\luku{ympyra-suora}{Ympyrä ja suora}
	% suoran ja ympyrän leikkauspisteet
	% tangentit
	\luku{paraabeli}{Paraabeli}
	% merkitys toisen asteen polynomin kuvaajana
	% geometrisen määritelmän maininta
	\luku{paraabeli-sovelluksia}{Paraabelin sovelluksia}
	% huipun x-koordinaatti on -b/2a
		% todistus liitteeksi tai tähän
	% paraabelin tangentit
	\luku{paraabeli-kaannetty}{Vasemmalle ja oikealle aukeavat paraabelit}
	% paraabeli x = ay^2  +by + c

\osa{Kertausosio}
	\nluku{LIITE_harjoituskokeita}{Harjoituskokeita}
	\nluku{LIITE_ylioppilaskokeita}{Ylioppilaskoetehtäviä}

\renewcommand{\luku}[2]{\section{#2} \lukufilter{#1}{\input{maa6/TEORIA_#1} \input{maa6/TEHT_#1}}} % luku
\renewcommand{\nluku}[2]{\section*{#2} \addcontentsline{toc}{section}{#2} \lukufilter{#1}{\input{maa6/#1}}} % numeroimaton luku

\newpage
\nosa{MAA6 -- Derivaatta}
\lukufilter{#1}{\subsection*{Lisälukemistoa}

Helsingin yliopiston perinteinen oppimateriaali:

\qrlinkki{http://www.math.helsinki.fi/kurssit/difint11/1999/all.pdf}{Hurri-Syrjänen, Ritva: Differentiaali- ja integraalilaskenta I.1}
}

\osa{Rationaalifunktiot}
	\luku{rationaalifunktiot}{Määritelmä}

\osa{Raja-arvo}
	\luku{raja-arvo-i}{Johdanto}
	\luku{raja-arvo-ii}{Raja-arvosäännöt II}
	\luku{jatkuvuus}{Jatkuvuus}

\osa{Derivaatta}
	\luku{derivaatta}{Johdanto}

\renewcommand{\nluku}[2]{\section*{#2} \addcontentsline{toc}{section}{#2} \lukufilter{#1}{\input{extra/#1}}} % numeroimaton luku

\osa{Syventäviä aiheita}

\nluku{LIITE_tasokayra}{Yleinen toisen asteen tasokäyrä}
	% esim. piste, kaksi suoraa, tyhjä
\nluku{LIITE_ellipsi}{Ellipsi}
\nluku{LIITE_hyperbeli}{Hyperbeli}
\nluku{LIITE_kolmioepayhtalo}{Kolmioepäyhtälö}
\nluku{LIITE_todistuksia}{Todistuksia}
	% kaava pisteen etäisyydelle suorasta
	% kaikkien paraabelien yhdenmuotoisuus
