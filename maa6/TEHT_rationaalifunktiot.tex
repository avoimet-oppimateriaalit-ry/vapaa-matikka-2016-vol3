\begin{tehtavasivu}

\begin{tehtava}
	Ratkaise epäyhtälö $|\frac{x^2+1}{2x+5}| < \frac15$
	\begin{vastaus}
		$0 < x < \frac25$
	\end{vastaus}
\end{tehtava}

\begin{tehtava}
% Laatinut Sampo Tiensuu 2014-01-09
Sievennä seuraavat rationaalilausekkeet:
\alakohdat{
    § $x+\frac{x}{1}+\frac{1}{x}$
    § $1:(x+1)+1$
    § $-\frac{7+x^2}{5}-\frac{2+x}{x}$
    § $\frac{2x+2x}{5x^2+\frac{x}{x+1}}$
    § $2x/5-3/5$	
%   § $\frac{ax+by}{cx^2+\frac{x}{x+1}}$
}

\begin{vastaus}
\alakohdat{
    § $\frac{2x^2+1}{x}$
    § $\frac{x+2}{x+1}$
    § $\frac{x^3-13x-10}{5x}$
    § $\frac{4x^2+4}{5x^2+5x+1}$
    § $\frac{2x-3}{5}$
}
\end{vastaus}
\end{tehtava}

\begin{tehtava}
Kahden polynomin summa, erotus ja tulo ovat aina polynomeja. Todista tähän vedoten, että kahden rationaalifunktion summa, tulo, erotus ja osamäärä ovat aina rationaalifunktioita.
\begin{vastaus}
Vinkki: Lavenna
\end{vastaus}
\end{tehtava}

\end{tehtavasivu}
